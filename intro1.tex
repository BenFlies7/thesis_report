http://www.nature.com/onc/journal/v26/n15/full/1210302a.html

Approximately half of all colorectal cancers show p53
(TP53) gene mutations, with higher frequencies observed in distal colon
and rectal tumors and lower frequencies in proximal tumors and those with
the microsatellite instability or methylator phenotypes. Alterations to
this gene appear to have little or no prognostic value for colorectal
cancer patients treated by surgery alone, but are associated with worse
survival for patients treated with chemotherapy. Several in vitro, animal
and clinical studies have shown that normal p53 is required for the
response of colorectal cancers to 5-fluorouracil-based chemotherapy.
http://www.ncbi.nlm.nih.gov/pubmed/12619112
http://www.ncbi.nlm.nih.gov/pubmed/25901683

Also in NSCLC: http://www.hindawi.com/journals/bmri/2011/583929/
http://mcr.aacrjournals.org/content/12/1/3.abstract
http://jnci.oxfordjournals.org/content/95/13/961.full
http://www.omicsonline.org/tp53-is-a-mutational-target-in-non-small-cell-lung-cancer-patients-2157-2518.1000138.php?aid=11703


Also in melanoma: http://www.ncbi.nlm.nih.gov/pubmed/7597286
http://bmccancer.biomedcentral.com/articles/10.1186/1471-2407-11-203
http://p53.free.fr/Database/p53_cancer/Human_Mutation_p53/p53_Skin.pdf





http://www.tandfonline.com/doi/pdf/10.4161/cc.5.18.3277


















\section{Introduction}

  Cancer represents a huge burden for health care systems worldwide and one of
  the leading death causes. In 2012, there were an estimated 14.1 million new
  cancer cases with estimated  8.2 million cancer deaths
  {\cite{cancer_stats_worldwide:2012}}. Lung cancer is the most common cancer,
  both in terms of new cases (1.8 million) and deaths (1.6 million). Breast
  cancer is the second most common cancer (1.7 million cases) but only ranks 5th
  as cause of death (522,000 deaths). Colorectal cancer (1.4 million cases;
  694,000 deaths), prostate cancer (1.1 million cases; 307,000 deaths), stomach
  cancer (951,000 cases; 723,000 deaths) and liver cancer (782,000 cases;
  723,000 deaths) are following.

  Scientific discoveries in the last decade have had an enormous impact on our
  understanding of the underlying causes of cancer. The development of omics
  techniques, in combination with enhanced computational power, has lead to an
  explosion of biological data. It has become clear that cancer is an incredibly
  complex malignancy. The research community is trying to interprete this vast
  amount of data with the goal to get a deeper understanding of cancer and to
  cure it eventually. In recent years, several drugs have been approved that
  target proteins needed for cancer development, proliferation or metastasis.
  Molecular testing is employed to check whether these targeted drugs would be
  of benefit. In that regard, Next-Generation Sequencing (NGS) is an interesting
  method to gain deep insights into the genetic information of a tumor and to
  guide personalized therapy.

  \subsection{Targeting Cancer}

    Cancers are characterized by several properties known as cancer hallmarks, which
    comprise blablabla

    In the last seventy years, five main models have dominated cancer research:

    \begin{itemize}
      \item The discovery of the carcinogenic potential of tobacco smoke lead to
        the the first and oldest model ('mutational'): long-term exposure to chemical
        carcinogens such as PAH or tobacco smoke induces cancerogenesis. The
        importance of viral DNA sequences and mutations induced by bacteria is also
        covered by this first model. In the induced mutational model, cancer is
        caused by external factors.

      \item The second model ('genome instability') puts emphasis on genome
        integrity, tumor-suppressor genes and oncogenes and DNA mismatch repair.
        Genomic instability can affect the genetic information at the level of
        nucleotides, microsatellites, whole genes or chromosomes. For instance,
        chromosome rearrangements, chromosome number alterations, loss of
        heterozygosity (LOH) or gene amplification contribute to chromosomal
        instability (CIN), which occurs in 50--85\% of colorectal cancers (CRCs).
        Even though numerous somatic mutations occur in tumors, only a small subset
        contributes to tumor progression. These 'driver' mutations

      \item Third model ('non-genotoxic') emphasizes on several important factors
        of cancer risk (obesity, diet, activity level, hormones). These factors do
        not induce structural changes, but rather act by inducing functional
        changes through epigenetic events such as histone methylation or DNA
        acetylation. Cancer then develops through selective selection of cells
        that have acquired a proliferative advantage through these processes.

      \item The fourth model ('Darwinian') is also based on clonal expansion, but
        puts emphasis on the macro- and micro-environment.

      \item Fifth model ('tissue organization'):

    \end{itemize}

    Cancer has been traditionally typified by a stepwise accumulation of
    mutations in key oncogenes and tumor suppressors. For decades, accumulation
    of these traits in somatic cells has been considered as the foundation of a
    developmental model of tumor progression where cells transition from a
    normal, healthy state to pre-malignant, malignant, and migratory phenotypes.

    Meanwhile, tumors are often described as heterogeneous, owing to the
    intricate genetic diversity and assorted morphological phenotypes they
    embody [2]. Intratumor heterogeneity specifically refers to heterogeneity
    within a tumor, while intertumor heterogeneity refers to heterogeneity
    across several different tumors [3]. The current view of tumor heterogeneity
    recognizes basic principles of Darwinian evolution at the core of neoplastic
    development and outgrowth: a single somatic cell with a heritable
    fitness-promoting mutation proliferates, conferring a survival advantage
    that allows cells to outlast the less ‘fit’ cells [3, 4]. Natural selection
    leads to sequential waves of clonal expansion, resulting in various
    subclones with differing capacities for proliferation, migration, and
    invasion [5].

    Advances in next-generation sequencing techniques and the inception of The
    Cancer Genome Atlas (TCGA) have revealed extensive heterogeneity at the
    molecular level [8]. Genetic heterogeneity of tumors is rooted in one of the
    key hallmarks of cancer: genetic instability [2]. Several mechanisms are in
    place in normal cells that protect against chromosome and nucleotide damage
    by preventing DNA replication until damage is repaired; however, genes
    controlling these critical checkpoints (e.g. p53) are often perturbed in
    cancer cells [16]. Genetic instability in cancer has been demonstrated at
    both the nucleotide level in point mutations and chromosome level in
    translocations, deletions, amplifications, and complete chromosome
    aneuploidy [17].

    Tumor cells undergo a series of genetic events that contribute to genomic
    instability throughout tumor progression (Figure 2A). However, the specific
    mechanisms and precise order in which they occur have yet to be elucidated
    [21]. Studies have pursued these mechanisms and found that the rate at which
    mutations occur in somatic cells is insufficient to cause the striking
    number of mutations present in cancer genomes. Over the past few decades, a
    ‘mutator’ hypothesis tumor evolution has emerged, speculating that a mutator
    phenotype characterized by genomic instability drives multi-step
    carcinogenesis and explaining the mutation rate discrepancy observed in
    normal and malignant cells [22].  The current mutator hypothesis speculates
    that a small number of ‘driver’ alterations exist and, once acquired by
    somatic mutation, confer the cancer phenotype; however, seemingly
    insignificant ‘passenger’ mutations result via mechanisms yet to be
    elucidated [26]. McFarland et al. challenged this with stochastic simulation
    of tumor evolution and reasoned that, though individually weak, the
    cooperative burden of small-scale accumulated passenger mutations has a
    present role in tumor progression, and may be the cause for complex
    oncological events that remain unanswered by the driver-centric model [27].

    Virtually all cancers tend to accumulate mutations during their progression.
    Studies have demonstrated that a typical cancer genome comprises about
    40--80 amino acid changing mutations. Some of these mutation increase the
    cancer's 'fitness' over that of  surrounding cells. The term 'fitness'
    here is defined by the difference between cell proliferation and cell death
    (net replication rate). Mutations that enhance the selective proliferative
    advantage of the cancer cell are called 'driver' mutations. It is often
    difficult to separate 'driver' from 'passenger' mutations, especially
    for low frequency variants. Passenger mutations occur in cancer cells
    subsequently or coincidentally to driver mutations and do not alter the cell's
    fitness. Typical solid tumors may contain 40--80 amino acid changing mutations,
    but only 5--15 of them are driver mutations.

    Which genetic aberrations? Mutational inactivation of tumor--suppressor genes,
    activation of oncogene pathways, an dann vun do op solid tumors, a rem
    zreck bei EGFR pathway, den erklären an targeted drugs erklären.

    Problem: heterogeneity & resistances

  \subsection{Targeting the EGFR Pathway in Solid Tumors}

    EGFR Pathway

    Targeted Drugs

    Resistances

  \subsection{Targeted Sequencing}

    Next generation sequencing can facilitate personalized cancer therapy
    approaches by identifying actionable so- matic events in tumor samples (1).
    Furthermore, high- quality sequencing data can reveal associations with sen-
    sitivity or resistance that can inform the development and implementation of
    targeted therapeutics Whole genome se- quencing (WGS) and whole exome
    sequencing (WES) allow the detection of SNVs, indels, CNVs, and rear-
    rangements. However, the relatively low coverage of WGS and WES, as
    currently implemented in most of the sequencing laboratories (100–250 ), may
    have limited ability to cost-effectively detect aberrations that are pres-
    ent in a subpopulation of tumor cells while identifying a myriad of
    aberrations of unknown clinical significance (2 ). Somatic aberrations
    present at low allele frequencies across different types of tumors (3, 4 )
    can potentially impact patient prognosis or response (5) and thus are
    important to detect reliably. Targeted sequencing to a depth that allows
    detection of relatively low mutant allele frequency (MAF) may represent an
    alternative or a com- plement to WGS and WES to detect clinically relevant
    alterations. Additionally, in most clinical and research settings, the
    amount of DNA that can be isolated from tumor samples is limited and the DNA
    is often damaged owing to fixation and storage procedures such as those used
    with formalin-fixed paraffin-embedded (FFPE) samples. Therefore a
    multiplexed targeted platform that can generate reliable data with high
    sensitivity from lim- ited amounts of DNA from FFPE samples is needed.
    Several targeted sequencing panels have been successfully implemented (6, 7 ).
    However, the details of a platform’s design and parameterization will
    influence the precision and reliability of the molecular profiling results,
    impact- ing both translational research and clinical decision- making. Thus,
    it is of great value to explore multiple potential solutions in a real
    patient care environment until a community-wide solution is established,
    vali- dated, and well accepted.

    \subsubsection{Target Enrichment Methods}

    \subsubsection{Illumina Sequencing Chemistry}

  \subsection{NGS Data Analysis}

    \subsubsection{GATK Best Practices}

  \subsection{Practical Implications in the Laboratory}

  \subsection{Aims of the Thesis}
