\section{Introduction}

  Cancer represents a huge burden for health care systems worldwide and one of
  the leading death causes. Scientific discoveries in the last decade have had
  an enormous impact on our understanding of the underlying causes of cancer.
  The development of omics techniques, in combination with enhanced
  computational power, has lead to an explosion of biological data. It has
  become clear that cancer is an incredibly complex malignancy. The research
  community is trying to interprete this vast amount of data with the goal to
  get a deeper understanding of cancer and to cure it eventually. In recent
  years, several drugs have been approved that target proteins needed for cancer
  development, proliferation or metastasis. Molecular testing is employed to
  check whether these targeted drugs would be of benefit. In that regard,
  Next-Generation Sequencing (NGS) is an interesting method to gain deep
  insights into the genetic information of a tumor and to guide personalized
  therapy.

  \subsection{Targeting Solid Tumors}

    \textbf{Melanoma}


    \textbf{Non-Small Cell Lung Carcinoma:} Lung cancer is the most common
    cancer in developed countries. Smoking is a widely accepted risk factor, as
    chemical carcinogens in tobacco smoke induce several genetic mutations.
    Oncogenic triggers cause cells of the normal bronchial epithelium to
    proliferate, giving rise to meta--, hyper-- and dysplastic epithelial
    lesions. blablabla


    \textbf{Colorectal Cancer}


    Virtually all cancers tend to accumulate mutations during their progression.
    The genetic diversity caused by this instability, the cardinal feature of
    cancer, in combination with several environmental factors, such as
    inflammation, enables the hallmarks of cancer (Hallmarks of Cancer: The Next
    Generation). These include replicative immortality, cell death resistance,
    ongoing proliferative signaling, invasion and metastasis, growth suppressor
    evasion, inducement of angiogenesis, energy metabolism reprogramming and
    immune destruction evasion.

    Large-scale studies have demonstrated that cancers are highly heterogeneous.
    This heterogeneity has been observed both at the inter- and intra-level. Two
    tumors of similar phenotype often comprise a different subset of mutations
    that may even have a low overlap. Several clonal subpopulations within the
    same tumor contribute to the intra-tumor heterogeneity. This is a
    considerable problem in the clinics, as some subpopulations of a cancer may
    become resistant to the  treatment and may be the source of relapses.

    Many alterations found in cancer cells are passenger mutations, e.g. do not
    contribute to the selective fitness of the cell. Driver mutations, often
    happening on oncogenes or tumor suppressor genes, promote the cell's
    fitness. This concept recognizes Darwinian evolution principles. The
    heterogeneous  population of cancer cells harbors cells with different
    random somatic and non-deleterious mutations and exhibits different
    perturbations. Cells with the best fitness, e.g. with the highest
    proliferative potential  and the lowest death rate, are then selected
    through natural selection principles. These cells will outlast less fit
    cells. This results in  sequential waves of clonal expansion, leading to
    different subclones within the same tumor that differ in their
    proliferative, migrative and invasive potential. The hypothesis that
    passenger mutations, that occur subsequently or coincidentally to driver
    mutations, do not influence the cell's fitness at all has been
    challenged by stochastic tumor evolution simulations (citat). Since then,
    it has been proposed that, even though the individual effect may be
    small, the cooperation of multiple accumulated small-scale passenger
    mutations plays a present role in cancer development and progression.

    Genomic instabilities: CIN, MSI, CpG.

    \subsubsection{Tumor Suppressors and Oncogenes in Solid Tumors}

      Genomic instability in cancerous cells becomes a critical mechanism if it
      affects oncogenes or tumor suppressor genes. Tumor suppressor genes protect
      a cell from entering the path to cancer. They comprise
      genes encoding for cell adhesion proteins,  DNA repair proteins, proteins
      acting in apoptosis pathways, or or cell cycle proteins. The action of these
      proteins inhibits metastasis, excessive cell survival or proliferation. Tumor
      suppressors mostly follow the two-hit hypothesis, which was first proposed
      by Knudson  for the retinoblastoma protein (pRb): to inactivate the
      tumor-protecting role of tumor suppressors, two genetic events,
      often LOH in  combination with silencing point mutations, are necessary to
      inactivate both alleles of the gene. Another possibility of tumor
      suppressor inactivation is methylation of the gene promoter. Compared to
      dominant oncogenes, tumor suppressor genes are often considered to be
      recessive. Alternatively, tumor progression can be influenced by functional
      haploinsufficiency. According to this conception, a disease state can emerge
      if a cell / organism has only one functional  copy of a given gene and if it
      cannot  produce enough of a gene product to establish a wild-type condition.
      Oncogenes comprise several GTPases, transcription factors, receptor tyrosine
      kinases and growth factors. Overexpressed or overactive versions of these
      proteins often lead to increased mitogenic signals, causing increased
      cell growth or proliferation. Mutations in proto--oncogenes can cause a
      loss of regulation or overactive proteins. Gene duplications or other
      chromosomal alterations lead to increased protein synthesis. Other
      mechanisms of importance include post-transcriptional mechanisms as
      misregulation of protein expression or increase of mRNA / protein stability.

      \textbf{APC:} Adenomatous Polyposis Coli (APC) gene codes for a 312 kDa
      protein. This multi--domain protein has binding sites for microtubules,
      cytoskeletal regulator proteins (IQGAP1, EB1) and Wnt signaling proteins
      (\beta--catenin, axin). 60\% of APC mutations in cancer present a
      C-terminal truncation, resulting in a loss of \beta--catenin and
      microtubule binding sites. Wnt signaling regulates, amongst others, cell
      migration, polarity, differentiation, adhesion and apoptosis. In the
      canonical Wnt signaling pathway, a destruction complex, including axin,
      GSK3, CK1\alpha, PP2A and APC, leads to \beta--catenin phosphorylation,
      followed by ubiquitination, marking it for degradation in the proteasome.
      Additionally, transcription factors of the TCF/LEF family form a complex
      with factors such as Groucho and histone deacetylases. This complex binds
      to Wnt signaling target genes and thereby represses gene expression. Once
      Wnt binds to the N-terminus of a cell surface receptor of the Frizzled
      family of receptors and and a co-receptor of the LRP5/6 family, the
      destruction complex is inhibited. Consequently, \beta--catenin is no
      longer marked for degradation and can translocate to the nucleus. There it
      displaces the factors binding to TCF/LEF and forms a complex with TCF/LEF,
      leading to activation of gene expression of target genes. Loss or
      dysfunction of APC leads to \beta--catenin accumulation in the nucleus
      even in the absence of  an extracellular stimulus.

      APC mutations are suspected to be the initiating event in many CRCs. APC
      mutations are sufficient for the growth of benign colorectal tumors.

      http://www.wormbook.org/chapters/www_wntsignaling/wntsignaling.html
      http://www.ncbi.nlm.nih.gov/pubmed/15768032
      http://jcs.biologists.org/content/120/19/3327.long
      http://www.ncbi.nlm.nih.gov/pmc/articles/PMC2634250/

      Also in melanoma: http://www.ncbi.nlm.nih.gov/pubmed/15133491

      Present, but less in NSCLC: http://www.ncbi.nlm.nih.gov/pubmed/15072829

      \textbf{TP53:} TP53 is one of the master guardians of the genome. In
      normal situations, p53, the protein encoded by TP53, is regulated by MDM2,
      MDM4 and E3--ubiquitin ligase, which target p53 for ubiquitination and
      degradation in the proteasome. In case of cellular stress, p53 is no
      longer ubiquitinated. p53 becomes activated in several situations, which
      include  DNA damage, osmotic shock, oxidative stress or oncogene
      expression. In such situations, p53 can then stop the cell cycle at the
      G1/S and G2/M transitions, induce DNA repair, and induce apoptosis if the
      damage cannot be repaired. TP53 thereby maintains genomic stability.

      One mechanism by which p53 acts on cell-cycle arrest is by activating
      expression of p21. p21 binds to the G1/S transition complex, formed by
      CDK4/CDK6, CDK2, CDK1) and inhibits its activity, leading to cell-cycle
      arrest. Inactivation or mutation of TP53 is a crucial step in many
      cancers. A defective p53 does not bind efficiently to DNA, resulting in
      less p21 expression. As a consequence, p21 cannot act as a cell-cycle stop
      signal.

      The importance of TP53 as tumor suppressor gene becomes evident in the
      autosomal dominant Li--Fraumeni syndrome. People suffering from this disorder inherit only
      one functional copy of TP53 and are likely to develop cancer in early
      ages.

      TP53 mutations are found in 50\% of CRCs,
      especially those associated with the methylator phenotype and
      microsatellite instability. Alterations in TP53 are associated with poor
      prognosis if treated with chemotherapy. In fact, wild-type TP53 is
      required for treatment with chemotherapy based on 5-fluoroacil.

      TP53 alterations are the early events in lung carcinogenesis


      \textbf{TGF--\beta:}
      Melanoma: http://www.ncbi.nlm.nih.gov/pmc/articles/PMC3662904/
      http://www.ncbi.nlm.nih.gov/pubmed/18426405
      http://www.ncbi.nlm.nih.gov/pubmed/21619542

      NSCLC: http://www.ncbi.nlm.nih.gov/pubmed/20107423
      http://link.springer.com/chapter/10.1007/978-1-4419-6615-5_28#page-1
      http://www.nature.com/cdd/journal/v21/n8/full/cdd201438a.html

      Colorectal: http://www.ncbi.nlm.nih.gov/pubmed/20517689
      http://www.ncbi.nlm.nih.gov/pmc/articles/PMC3512565/
      http://hmg.oxfordjournals.org/content/16/R1/R14.full

      \textbf{RAS--RAF--MAPK:}

      RAS-RAF-MAPK induces cell survival and cell cycle progression and
      proliferation

      \textbf{PI3K (Phosphatidylinositol 3--Kinase:}

      Activation of
      PTEN/PI3K/AKT leads to cell growth, proliferation and survival
      http://www.nature.com/onc/journal/v27/n41/full/onc2008247a.html
      http://www.ncbi.nlm.nih.gov/pmc/articles/PMC3092286/
      http://www.hindawi.com/journals/isrn/2013/472432/

      \paragraph{EGFR Signaling Pathway}

https://www.moffitt.org/File%20Library/Main%20Nav/Research%20and%20Clinical%20Trials/Cancer%20Control%20Journal/v14n3/295.pdf
http://www.sciencedirect.com/science/article/pii/S0014579301021664
http://www.ncbi.nlm.nih.gov/pubmed/18038764
http://www.ncbi.nlm.nih.gov/pubmed/12517767

      The EGFR signaling pathway is frequently altered in
      many cancers. EGFR signaling acts through the RAS-RAF-MAPK and PI3K-Akt pathways.
      EGFR is part of the family of receptor tyrosine kinases. This transmembrane
      protein is composed of an intracytoplasmic tyrosine kinase domain, a short
      hydrophobic transmembrane region and an extracellular ligand-binding domain.
      Ligand (EGF, TGFα) binding causes homo-- or heterodimerization, followed by
      an auto-- and cross--phosphorylation of key tyrosine residues on its
      cytoplasmic domain. This forms docking sites for cytoplasmic proteins that
      contain phosphotyrosine-binding and Src homology 2 domains.

      Many cancers have been shown to be dependent on EGFR signaling: they
      overexpress EGFR or to harbor activating mutations in EGFR or downstream
      proteins, both leading to increased mitogenic signals. Targeting the EGFR
      signaling pathway is an attractive target blabla and has been of benefit in
      solid tumors, e.g. melanoma, CRC and NSCLC. Advantages have been made better
      survival rates. shutting egfr down -> apoptosis. but 2 problems: resistances &
      not all mutations are actionable. therefore: molecular testing. the more
      comprehensible, the better. classical approaches only target some hotspot
      regions. NGS has the potential to give really deep insights

      In recent years, several EGFR targeted anti-cancer drugs have been
      approved by the FDA. Anti--EGFR targeted monoclonal antibodies and
      EGFR--specific tyrosine--kinase inhibitors have shown their usefulness in
      the treatment of solid tumors.

      However, solid tumors have a tendency to harbor mutations in proteins
      acting in  the EGFR signaling pathway. Table XXX shows the frequency of
      tumors  harboring mutations in EGFR or downstream proteins. Identifying
      the mutational status of these proteins is of utmost importance in
      targeted anti-EGFR therapy. Wild-type or mutated proteins provide either
      increased sensitivity or resistance to the treatment.

      predictive mutations

      A common target of somatic mutations, especially at codons 12 (82–87%), 13
      (13–18%), and 61, KRAS has been implicated in many human cancers [46].
      KRAS mutations have been reported in 40% of CRCs and contribute to the
      development of colorectal adenomas and hyperplastic polyps [2]. These
      mutations are usually single nucleotide point mutations that lock the
      enzyme bound to ATP, by inhibiting its GTPase activities thus upreg-
      ulating the Ras function [13, 22]. It, therefore, affects downstream
      signalling cascades including MAPK and PI3K. Early KRAS mutations have
      been identified in left-sided hyperplastic polyps [10]. This mutation is
      more common in polypoid lesions than nonpolypoid [6]. KRAS mutations are
      associated with a worse prognosis, in part due to the overexpression of
      KRAS contributing to metastases through increasing the production of
      protease to degrade the ex- tracellular matrix.

      KRAS mutations have recently gained interest as a negative predictive
      factor for anti-EGFR therapy response. Blocking EGFR will have no effect
      if KRAS is mutated as it functions downstream of EGF receptors. Thus, a
      KRAS mutation allows continual activation of the downstream pathway, thus
      negating the effects of the drug [41]. As such, anti-EGFR drugs (Section
      7.1.2) are not recommended in KRAS-mutated tumours. In this context, it
      has been suggested that all patients with CRC under consideration for
      anti-EGFRs should be tested for KRAS mutation status prior to treatment
      initiation [16, 41].

      The Raf family includes three members: ARAF, RAF1, and BRAF. When
      activated, these serine/threonine kinases activate MEK1 and MEK2 which
      phosphorylate ERK1 and ERK2. The ERKs continue the cascade by
      phosphorylating cytosolic and nuclear substrates such as JUN and ELK1 that
      regulate a wide spectrum of enzymes such as cyclin D1 [13]. Similar to
      KRAS, BRAF mutations render it continually active, in over 80% of CRCs by
      substitution of thymine to adenine at nucleotide 1799 that results in a
      substitution of valine to glutamic acid [5]. These point mutations make
      BRAF an attractive marker for analysis, as they are present in at least
      80% of mutants [22]. Such mutations are more frequent in MSI tumours and
      are reported in 5–15% of CRCs [5]. Mutations in BRAF and KRAS are mutually
      ex- clusive as they are intimately connected in the RAS-RAF- MAPK pathway
      [45].

      BRAF mutations confer a worse clinical outcome and thus the need for
      adjuvant therapy [5]. Mutations are asso- ciated with a shorter
      progression free and overall survival [45]. Though controversial, some
      studies have found that these adverse clinical effects of BRAF are negated
      in CIMP+ tumours, suggesting the poor prognosis is not attributable to the
      BRAF mutation itself, but is probably attributable to the genetic pathway
      in which it occurs [5]. Similar to KRAS, BRAF mutations have also been
      implicated in anti-EGFR resistance. Approximately 60% of KRAS wild-type
      metastatic CRC (mCRC) are unresponsive to these drugs, and it is
      hypothesized that BRAF mutations may confer some of this resistance [41].
      As such, BRAF mutation status may also be assessed to determine patients
      resistant to anti-EGFR therapy.

      Targeted cancer therapy holds the promise of highly selective
      tumor cell killing while sparing most of normal proliferating cells,
      thus avoiding some side effects of conventional cytotoxic therapy.

    \begin{table}[!htbp]
        \caption[Occurrence of mutations]{EGFR signaling pathway components affected in colorectal cancer, melanoma and non-small cell lung carcinoma}
        \centering
        \begin{tabular}{ |p{2cm}|p{2cm}|p{2cm}|p{2cm}|}
        \hline
        Gene & CRC (\%) & Melanoma & NSCLC \\ \hline \hline
        EGFR & NA & NA & 10--35 \\
        KRAS & 36--40 & 2 & 15--25 \\
        NRAS & 1--6 & 13--25 & 1 \\
        BRAF & 8--15 & 37--50 & 1--3 \\
        PTEN & 5--14 & NA & 4--8 \\
        PIK3CA & 10--30 & & 1--3 \\
        \hline
      \end{tabular}
    \end{table}
(http://www.nature.com/bjc/journal/v112/n2/full/bjc2014476a.html)

    \begin{table}[!htbp]
        \caption[Targeted Cancer Agents]{FDA-approved cancer drugs for solid tumor treatment that target the EGFR pathway}
        \centering
        \begin{tabular}{ |p{4cm}|p{3.7cm}|p{6.3cm}|}
        \hline
        Agent & Target(s) & FDA-approved indication(s) \\ \hline \hline
        Afatinib (Gilotrif) & EGFR & NSCLC (with EGFR del19 or L858R) \\
        Cetuximab (Erbitux) & EGFR & Colorectal cancer (KRAS WT) \\
        Cobimetinib (Cotellic) & MEK & Melanoma (with BRAF V600E or V600K \\
        Dabrafenib (Tafinlar) & BRAF & Melanoma (with BRAF V600 mutation) \\
        Erlotinib (Tarceva) & EGFR & NSCLC \\
        Gefitinib (Iressa) & EGFR & NSCLC (with EGFR del19 or L858R) \\
        Necitumumab (Portrazza) & EGFR & Squamous NSCLC \\
        Osimertinib (Tagrisso) & EGFR & NSCLC (with EGFR T790M) \\
        Panitumumab (Vectibix) & EGFR & Colorectal cancer (KRAS WT) \\
        Trametinib (Mekinist) & MEK & Melanoma (with BRAF V600) \\
        Vemurafenib (Zelboraf) & BRAF & Melanoma (with BRAF V600) \\
        \hline
      \end{tabular}
    \end{table}

  \subsection{Targeted Sequencing}

    \subsubsection{Target Enrichment Methods}

    \subsubsection{Illumina Sequencing Chemistry}

  \subsection{NGS Data Analysis}

    \subsubsection{GATK Best Practices}

  \subsection{Practical Implications in the Laboratory}

  \subsection{Aims of the Thesis}
