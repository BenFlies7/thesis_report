\section{Introduction}

  Cancer represents a huge burden for health care systems worldwide and one of
  the leading death causes. In 2012, there were an estimated 14.1 million new
  cancer cases with estimated  8.2 million cancer deaths
  {\cite{cancer_stats_worldwide:2012}}. Lung cancer is the most common cancer,
  both in terms of new cases (1.8 million) and deaths (1.6 million). Breast
  cancer is the second most common cancer (1.7 million cases) but only ranks 5th
  as cause of death (522,000 deaths). Colorectal cancer (1.4 million cases;
  694,000 deaths), prostate cancer (1.1 million cases; 307,000 deaths), stomach
  cancer (951,000 cases; 723,000 deaths) and liver cancer (782,000 cases;
  723,000 deaths) are following.

  Scientific discoveries in the last decade have had an enormous impact on our
  understanding of the underlying causes of cancer. The development of omics
  techniques, in combination with enhanced computational power, has lead to an
  explosion of biological data. It has become clear that cancer is an incredibly
  complex malignancy. The research community is trying to interprete this vast
  amount of data with the goal to get a deeper understanding of cancer and to
  cure it eventually. In recent years, several drugs have been approved that
  target proteins needed for cancer development, proliferation or metastasis.
  Molecular testing is employed to check whether these targeted drugs would be
  of benefit. In that regard, Next-Generation Sequencing (NGS) is an interesting
  method to gain deep insights into the genetic information of a tumor and to
  guide personalized therapy.

  \subsection{Targeting Cancer}

    Cancerogenesis is considered to be caused by an imbalance between the
    occurrence of mutations and cell-cycle control mechanisms, thereby leading
    to cardinal features of cancer: genomic instability and modifications. These
    alterations are caused either by inherited mutations or are acquired during
    cancerogenesis. Critical alterations include single nucleotide variations
    (SNVs), insertions and deletions of one or multiple nucleotides (INDELS),
    copy number variations (CNVs) and rearrangements.

    Genomic instability can affect the genetic information at the level of
    nucleotides, microsatellites, whole genes or chromosomes. For instance,
    chromosome rearrangements, chromosome number alterations, loss of
    heterozygosity (LOH) or gene amplification contributee to chromosomal
    instability (CIN), which occurs in 50--85\% of colorectal cancers (CRCs).
    Even though numerous somatic mutations occur in tumors, only a small subset
    contributes to tumor progression. These \"driver\" mutations

    Accumulation of mutations during cancer progression

    Genomic modifications in tumors lead to the inactivation of tumor-suppressor
    genes and / or the activation of oncogene pathways. The normal activity of
    tumor-suppressor genes acts to limit cancer development and proliferation
    by, for instance, controlling the cell cycle or cell motility. This
    protection is lost in many cancers through several mechanisms that include
    SNVs, deletions, loss of one allele with mutation on the other and promoter
    methylation. Genes affected by these modifications include TP53, APC and
    TGF--β. p53 and its coding gene TP53 are well-known to be master regulators,
    which are misregulated in many cancer types. The inactivation of TP53 by LOH
    and mutations is a crucial step in many CRCs.

    Oncogenes encode for signalling molecules, cell-cycle regulators, growth
    factors and their receptors. Epigenetic / transcriptional modifications,
    copy number amplifications or modifications that impair the normal function
    of the protein lead to overexpressed or overactive gene products, giving the
    cell, in which the alteration occurs, a proliferative advantage.

  \subsection{Targeting the EGFR Pathway in Solid Tumors}

    EGFR Pathway

    Targeted Drugs

    Resistances

  \subsection{Targeted Sequencing}

    \subsubsection{Target Enrichment Methods}

    \subsubsection{Illumina Sequencing Chemistry}

  \subsection{NGS Data Analysis}

    \subsubsection{GATK Best Practices}

  \subsection{Practical Implications in the Laboratory}

  \subsection{Aims of the Thesis}
