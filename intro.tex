\section{Introduction}

  Cancer represents a huge burden for health care systems worldwide and is one
  of the leading death causes. Scientific discoveries in the last decade have
  had an enormous impact on our understanding of the underlying causes of
  cancer. The development of omics techniques, in combination with advanced
  computational power, has lead to an explosion of biological data. It has
  become clear that cancer is an incredibly complex malignancy, which is
  affected by genetic, environmental and behavioural factors. The research
  community is trying to interprete this vast amount of data with the goal to
  get a deeper understanding of cancer and to cure it eventually. In recent
  years, several drugs have been approved, which target proteins needed for cancer
  development, proliferation or metastasis. Molecular testing is employed to
  check whether these targeted drugs would be of benefit. In that regard,
  Next-Generation Sequencing (NGS) is an interesting method to gain deep
  insights into the genetic information of a tumor and to guide personalized
  therapy.

  \subsection{Targeting Solid Tumors}

    \textbf{Melanoma} develops from the malignant transformation of melanocytes
    in the basal epidermal layer of the skin. Melanoma incidence has exploded
    over the last four decades, with a 15-fold increase in the United States.
    Both genetic predisposition and environmental factors influence the risk
    of getting melanoma. Skin cancer often affects fair-skinned individuals.
    Exposure to UV light, immunosuppression and multiple nevi are risk factors.
    UV radiation causes cyclobutane pyrimidine dimers (CPDs). By joining
    adjacent pyrimidine bases, T--T, C--C or C--T dimers (called UV
    fingerprints) are formed, leading to direct DNA damage. People diagnosed
    with rare genetic disorders like xeroderma pigmentosum are at great risk.
    Several susceptibility genes also increase melanoma risk, such as the MC1R
    gene that caused red hair.

    At stage 0, tumor cells are still limited to the epidermis. In the radial growth phase is the earliest stage of melanoma. The tumor has a
    thickness less than 1mm and the cancer cells have not yet reached blood
    vessels. The cancer cells then acquire invasive potential: the cancer enters
    the invasive radial growth phase. In the vertical growth phase, tumor cells
    enter the blood stream or lymph vessels. The tumor starts to grow into
    surrounding tissues and is has now a tickness more than 1mm. Survival rates
    and treatment options decrease drastically with each stage.

    Stage 0: Melanoma involves the epidermis but has not reached the underlying dermis.
    Stages I and II: Melanoma is characterized by tumor thickness and ulceration status. No evidence of regional lymph node or distant metastasis.
    Stage III: Melanoma is characterized by lymph node metastasis. No evidence of distant metastasis.
    Stage IV: Melanoma is characterized by the location of distant metastases and the level of lactate dehydrogenase.


    \textbf{Non-Small Cell Lung Carcinoma:} Lung cancer is the most common
    cancer in developed countries. Smoking is a widely accepted risk factor, as
    chemical carcinogens in tobacco smoke induce several genetic mutations.
    Oncogenic triggers cause cells of the normal bronchial epithelium to
    proliferate, giving rise to meta--, hyper-- and dysplastic epithelial
    lesions. blablabla


    \textbf{Colorectal Cancer}

    \subsubsection{Genomic Instability in Cancers}

      Virtually all cancers tend to accumulate mutations during their progression.
      The genetic diversity caused by this instability, the cardinal feature of
      cancer, in combination with several environmental factors, such as
      inflammation, enables the hallmarks of cancer (Hallmarks of Cancer: The Next
      Generation). These include replicative immortality, cell death resistance,
      ongoing proliferative signaling, invasion and metastasis, growth suppressor
      evasion, inducement of angiogenesis, energy metabolism reprogramming and
      immune destruction evasion.

      Large-scale studies have demonstrated that cancers are highly heterogeneous.
      This heterogeneity has been observed both at the inter- and intra-level. Two
      tumors of similar phenotype often comprise a different subset of mutations
      that may even have a low overlap. Several clonal subpopulations within the
      same tumor contribute to the intra-tumor heterogeneity. This is a
      considerable problem in the clinics, as some subpopulations of a cancer may
      become resistant to the  treatment and may be the source of relapses.

      Genomic instabilities: CIN, MSI, CpG.

      Many alterations found in cancer cells are passenger mutations, e.g. do not
      contribute to the selective fitness of the cell. Driver mutations, often
      happening on oncogenes or tumor suppressor genes, promote the cell's
      fitness. This concept recognizes Darwinian evolution principles. The
      heterogeneous  population of cancer cells harbors cells with different
      random somatic and non-deleterious mutations and exhibits different
      perturbations. Cells with the best fitness, e.g. with the highest
      proliferative potential  and the lowest death rate, are then selected
      through natural selection principles. These cells will outlast less fit
      cells. This results in  sequential waves of clonal expansion, leading to
      different subclones within the same tumor that differ in their
      proliferative, migrative and invasive potential. The hypothesis that
      passenger mutations, that occur subsequently or coincidentally to driver
      mutations, do not influence the cell's fitness at all has been
      challenged by stochastic tumor evolution simulations (citat). Since then,
      it has been proposed that, even though the individual effect may be
      small, the cooperation of multiple accumulated small-scale passenger
      mutations plays a present role in cancer development and progression.

    \subsubsection{Tumor Suppressors and Oncogenes in Solid Tumors}

      Genomic instability in cancerous cells becomes a critical mechanism if it
      affects oncogenes or tumor suppressor genes. Tumor suppressor genes protect
      a cell from entering the path to cancer. They comprise
      genes encoding for cell adhesion proteins,  DNA repair proteins, proteins
      acting in apoptosis pathways, or or cell cycle proteins. The action of these
      proteins inhibits metastasis, excessive cell survival or proliferation. Tumor
      suppressors mostly follow the two-hit hypothesis, which was first proposed
      by Knudson  for the retinoblastoma protein (pRb): to inactivate the
      tumor-protecting role of tumor suppressors, two genetic events,
      often LOH in  combination with silencing point mutations, are necessary to
      inactivate both alleles of the gene. Another possibility of tumor
      suppressor inactivation is methylation of the gene promoter. Compared to
      dominant oncogenes, tumor suppressor genes are often considered to be
      recessive. Alternatively, tumor progression can be influenced by functional
      haploinsufficiency. According to this conception, a disease state can emerge
      if a cell / organism has only one functional  copy of a given gene and if it
      cannot  produce enough of a gene product to establish a wild-type condition.
      Oncogenes comprise several GTPases, transcription factors, receptor tyrosine
      kinases and growth factors. Overexpressed or overactive versions of these
      proteins often lead to increased mitogenic signals, causing increased
      cell growth or proliferation. Mutations in proto--oncogenes can cause a
      loss of regulation or overactive proteins. Gene duplications or other
      chromosomal alterations lead to increased protein synthesis. Other
      mechanisms of importance include post-transcriptional mechanisms as
      misregulation of protein expression or increase of mRNA / protein stability.

      \textbf{APC:} Adenomatous Polyposis Coli (APC) gene codes for a 312 kDa
      protein. This multi--domain protein has binding sites for microtubules,
      cytoskeletal regulator proteins (IQGAP1, EB1) and Wnt signaling proteins
      (b -catenin, axin). 60\% of APC mutations in cancer present a
      C-terminal truncation, resulting in a loss of b--catenin and
      microtubule binding sites. Wnt signaling regulates, amongst others, cell
      migration, polarity, differentiation, adhesion and apoptosis. In the
      canonical Wnt signaling pathway, a destruction complex, including axin,
      GSK3, CK1a, PP2A and APC, leads to b--catenin phosphorylation,
      followed by ubiquitination, marking it for degradation in the proteasome.
      Additionally, transcription factors of the TCF/LEF family form a complex
      with factors such as Groucho and histone deacetylases. This complex binds
      to Wnt signaling target genes and thereby represses gene expression. Once
      Wnt binds to the N-terminus of a cell surface receptor of the Frizzled
      family of receptors and and a co-receptor of the LRP5/6 family, the
      destruction complex is inhibited. Consequently, b--catenin is no
      longer marked for degradation and can translocate to the nucleus. There it
      displaces the factors binding to TCF/LEF and forms a complex with TCF/LEF,
      leading to activation of gene expression of target genes. Loss or
      dysfunction of APC leads to b--catenin accumulation in the nucleus
      even in the absence of  an extracellular stimulus.

      APC mutations are suspected to be the initiating event in many CRCs. APC
      mutations are sufficient for the growth of benign colorectal tumors.

      http://www.wormbook.org/chapters/www_wntsignaling/wntsignaling.html
      http://www.ncbi.nlm.nih.gov/pubmed/15768032
      http://jcs.biologists.org/content/120/19/3327.long
      http://www.ncbi.nlm.nih.gov/pmc/articles/PMC2634250/

      Also in melanoma: http://www.ncbi.nlm.nih.gov/pubmed/15133491

      Present, but less in NSCLC: http://www.ncbi.nlm.nih.gov/pubmed/15072829

      \textbf{TP53:} TP53 is one of the master guardians of the genome. In
      normal situations, p53, the protein encoded by TP53, is regulated by MDM2,
      MDM4 and E3--ubiquitin ligase, which target p53 for ubiquitination and
      degradation in the proteasome. In case of cellular stress, p53 is no
      longer ubiquitinated. p53 becomes activated in several situations, which
      include  DNA damage, osmotic shock, oxidative stress or oncogene
      expression. In such situations, p53 can then stop the cell cycle at the
      G1/S and G2/M transitions, induce DNA repair, and induce apoptosis if the
      damage cannot be repaired. TP53 thereby maintains genomic stability.

      One mechanism by which p53 acts on cell-cycle arrest is by activating
      expression of p21. p21 binds to the G1/S transition complex, formed by
      CDK4/CDK6, CDK2, CDK1) and inhibits its activity, leading to cell-cycle
      arrest. Inactivation or mutation of TP53 is a crucial step in many
      cancers. A defective p53 does not bind efficiently to DNA, resulting in
      less p21 expression. As a consequence, p21 cannot act as a cell-cycle stop
      signal.

      The importance of TP53 as tumor suppressor gene becomes evident in the
      autosomal dominant Li--Fraumeni syndrome. People suffering from this disorder inherit only
      one functional copy of TP53 and are likely to develop cancer in early
      ages.

      TP53 mutations are found in 50\% of CRCs,
      especially those associated with the methylator phenotype and
      microsatellite instability. Alterations in TP53 are associated with poor
      prognosis if treated with chemotherapy. In fact, wild-type TP53 is
      required for treatment with chemotherapy based on 5-fluoroacil.

      TP53 alterations are the early events in lung carcinogenesis


      \textbf{TGF--b:}
      Melanoma: http://www.ncbi.nlm.nih.gov/pmc/articles/PMC3662904/
      http://www.ncbi.nlm.nih.gov/pubmed/18426405
      http://www.ncbi.nlm.nih.gov/pubmed/21619542

      NSCLC: http://www.ncbi.nlm.nih.gov/pubmed/20107423
      http://link.springer.com/chapter/10.1007/978-1-4419-6615-5_28#page-1
      http://www.nature.com/cdd/journal/v21/n8/full/cdd201438a.html

      Colorectal: http://www.ncbi.nlm.nih.gov/pubmed/20517689
      http://www.ncbi.nlm.nih.gov/pmc/articles/PMC3512565/
      http://hmg.oxfordjournals.org/content/16/R1/R14.full

      \paragraph{EGFR signaling pathway}: EGFR is a protein of the tyrosine kinase
      receptor family. It is anchored in the cytoplasmic membrane and is
      composed of an intracytoplasmic tyrosine kinase domain, a short
      hydrophobic transmembrane domain and an extracellular ligand-binding
      domain. Ligand binding causes a conformational change if the receptor,
      which leads to homo-- or heterodimerization, followed by an auto-- and
      cross--phosphorylation of key tyrosine residues on its cytoplasmic domain.
      This forms docking sites for cytoplasmic proteins that contain
      phosphotyrosine-binding and Src homology 2 domains. These proteins are
      adaptor molecules for the RAS-RAF-MAPK and PI3K pathways. Both pathways
      lead to cell survival, proliferation and invasion.

      PTEN/PI3K/AKT leads to cell growth, proliferation and survival
      http://www.nature.com/onc/journal/v27/n41/full/onc2008247a.html
      http://www.ncbi.nlm.nih.gov/pmc/articles/PMC3092286/
      http://www.hindawi.com/journals/isrn/2013/472432/

      Phospholipase C

      STATs

      Src

      In the RAS-RAF-MAPK pathway, GRB2 binds to Tyr1068 of EGFR through their
      SH2 domain and recruits SOS, a guanine nucleotide exchange factor. Grb2
      and SOS then form a complex with the activated EGFR, which activates SOS.
      Activates SOS, through its GEF activity, then induces GDP removal from Ras
      proteins, which can subsequently bind GTP and become active. Ras then
      activates Raf serine/threonine kinase proteins, which phosphorylate and
      thereby activate MEKs, which are tyrosine/threonine kinases. Activated
      MEKS then phosphorylate and activate MAPKs, also serine/threonine kinases.
      MAPKs then act on the expression of target genes that promote cell survival,
      cell cycle progression and proliferation.

      http://www.ncbi.nlm.nih.gov/pmc/articles/PMC3457779/

      The RAS-RAF-MAPK pathway is deregulated in many cancers, mainly through
      activating mutatios on RAS or RAF.

      \textbf{KRAS}
      KRAS has gained interest as negative predictive
      marker of the successfulness of anti-EGFR targeted therapy. KRAS is
      mutated in 36--40\% of CRCs, 15--25\% of NSCLCs and 2\% of melanomas.
      Single nucleotide point mutations in codons 12, 13 and 61 can act as
      activating mutations. Proteins affected by such mutations are locked in
      their active GTP-bound state and are consequently constantly active.

      \textbf{BRAF}
      Apart from RAS proteins, RAF proteins are of importance in solid tumors.

      https://www.moffitt.org/File%20Library/Main%20Nav/Research%20and%20Clinical%20Trials/Cancer%20Control%20Journal/v14n3/295.pdf
      http://www.sciencedirect.com/science/article/pii/S0014579301021664
      http://www.ncbi.nlm.nih.gov/pubmed/18038764
      http://cancerres.aacrjournals.org/content/63/1/1.long

      Many cancers have been shown to be dependent on EGFR signaling.Targeting
      the EGFR signaling pathway is an attractive target blabla and has been of
      benefit in solid tumors, e.g. melanoma, CRC and NSCLC. Advantages have
      been made better survival rates. shutting egfr down -> apoptosis. but 2
      problems: resistances & not all mutations are actionable. therefore:
      molecular testing. the more comprehensible, the better. classical
      approaches only target some hotspot regions. NGS has the potential to give
      really deep insights

      In recent years, several EGFR targeted anti-cancer drugs have been
      approved by the FDA. Anti--EGFR targeted monoclonal antibodies and
      EGFR--specific tyrosine--kinase inhibitors have shown their usefulness in
      the treatment of solid tumors.

      However, solid tumors have a tendency to harbor mutations in proteins
      acting in  the EGFR signaling pathway. Table XXX shows the frequency of
      tumors  harboring mutations in EGFR or downstream proteins. Identifying
      the mutational status of these proteins is of utmost importance in
      targeted anti-EGFR therapy. Wild-type or mutated proteins provide either
      increased sensitivity or resistance to the treatment.


      Targeted cancer therapy holds the promise of highly selective
      tumor cell killing while sparing most of normal proliferating cells,
      thus avoiding some side effects of conventional cytotoxic therapy.

    \begin{table}[!htbp]
        \caption[Occurrence of mutations]{EGFR signaling pathway components affected in colorectal cancer, melanoma and non-small cell lung carcinoma}
        \centering
        \begin{tabular}{ |p{2cm}|p{2cm}|p{2cm}|p{2cm}|}
        \hline
        Gene & CRC (\%) & Melanoma & NSCLC \\ \hline \hline
        EGFR & NA & NA & 10--35 \\
        KRAS & 36--40 & 2 & 15--25 \\
        NRAS & 1--6 & 13--25 & 1 \\
        BRAF & 8--15 & 37--50 & 1--3 \\
        PTEN & 5--14 & NA & 4--8 \\
        PIK3CA & 10--30 & & 1--3 \\
        \hline
      \end{tabular}
    \end{table}
(http://www.nature.com/bjc/journal/v112/n2/full/bjc2014476a.html)

    \begin{table}[!htbp]
        \caption[Targeted Cancer Agents]{FDA-approved cancer drugs for solid tumor treatment that target the EGFR pathway}
        \centering
        \begin{tabular}{ |p{4cm}|p{3.7cm}|p{6.3cm}|}
        \hline
        Agent & Target(s) & FDA-approved indication(s) \\ \hline \hline
        Afatinib (Gilotrif) & EGFR & NSCLC (with EGFR del19 or L858R) \\
        Cetuximab (Erbitux) & EGFR & Colorectal cancer (KRAS WT) \\
        Cobimetinib (Cotellic) & MEK & Melanoma (with BRAF V600E or V600K \\
        Dabrafenib (Tafinlar) & BRAF & Melanoma (with BRAF V600 mutation) \\
        Erlotinib (Tarceva) & EGFR & NSCLC \\
        Gefitinib (Iressa) & EGFR & NSCLC (with EGFR del19 or L858R) \\
        Necitumumab (Portrazza) & EGFR & Squamous NSCLC \\
        Osimertinib (Tagrisso) & EGFR & NSCLC (with EGFR T790M) \\
        Panitumumab (Vectibix) & EGFR & Colorectal cancer (KRAS WT) \\
        Trametinib (Mekinist) & MEK & Melanoma (with BRAF V600) \\
        Vemurafenib (Zelboraf) & BRAF & Melanoma (with BRAF V600) \\
        \hline
      \end{tabular}
    \end{table}

  \subsection{Targeted Sequencing}

    \subsubsection{Target Enrichment Methods}

    \subsubsection{Illumina Sequencing Chemistry}

  \subsection{NGS Data Analysis}

    \subsubsection{GATK Best Practices}

  \subsection{Practical Implications in the Laboratory}

  \subsection{Aims of the Thesis}
