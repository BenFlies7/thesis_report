\section{Introduction}

  Cancer represents a huge burden for health care systems worldwide and is one
  of the leading death causes. Scientific discoveries in the last decade have
  had an enormous impact on our understanding of the underlying causes of
  cancer. The development of omics techniques, in combination with advanced
  computational power, has lead to an explosion of biological data. It has
  become clear that cancer is an incredibly complex malignancy, which is
  affected by genetic, environmental and behavioural factors. The research
  community is trying to interprete this vast amount of data with the goal to
  get a deeper understanding of cancer and to cure it eventually. In recent
  years, several drugs have been approved, which target proteins needed for
  cancer development, proliferation or metastasis. Molecular testing is employed
  to check whether these targeted drugs would be of benefit. In that regard,
  Next-Generation Sequencing (NGS) is an interesting method to gain deep
  insights into the genetic information of a tumor and to guide personalized
  therapy.

  \subsection{The cancer genome}

    DNA undergoes continuous damage. In normal cells, this damage is repaired
    without without errors. In cancer cells, the equilibrium between DNA damage
    and repair systems is dysbalanced {\cite{dna_repair_epidemioloy}}, leading
    to a mutator phenotype. The resulting genomic instability manifests itself
    in an accumulation of mutations.

    The genetic diversity caused by this instability, the cardinal feature of
    cancer, in combination with several environmental factors, such as
    inflammation, enables the hallmarks of cancer {\cite{cancer_hallmarks}}.
    These include replicative immortality, cell death resistance, ongoing
    proliferative signaling, invasion and metastasis, growth suppressor evasion,
    inducement of angiogenesis, energy metabolism reprogramming and immune
    destruction evasion.

    \subsubsection{Cancer: an evolutionary process}

      Cancer progression is a process that recognizes basic Darwinian evolution
      principles {\cite{clonal_evolution}} {\cite{darwinian_models}}
      {\cite{war_zone}} {\cite{cancer_models}}. Similarly as proposed in Darwins
      origins of species,  cancer development and progression is based on two
      distinct processes.  First, the population of cells has to harbor
      heritable genetic variation. These mutations may be of germline origin or
      may occur through somatic processes. If the occurring mutations are
      non--deleterious, they can be passed on to the next generation of cells.
      The second process, which has to take place in Darwinian evolution is
      natural selection. Each cell exhibits a unique combination of genetic and
      environmental perturbations. Cells are in competition for a variety of
      resources in their microenvironment, which include space, oxygen and
      nutrients. Eventually, cells with the best fitness, e.g. with the highest
      proliferative potential and the lowest death rate, are then selected
      through natural selection principles. These cells will outlast less fit
      cells. This results in sequential waves of clonal expansion
      {\cite{clonal_evolution}}, leading to different subclones within the same
      tumor that differ in their proliferative, migrative and invasive
      potential.

    \subsubsection{Accumulation of somatic mutations}

      It is widely accepted that tumors accumulate somatic mutations during
      their progression in malignancy {\cite{accumulation_rates}}
      {\cite{mutations_counting}}. Somatic mutations can have distinct origins
      {\cite{multiple_mutations}}. DNA can be damaged by endogenous and
      environmental agents. Carcinogenic substances produced by industry
      {\cite{occupational_exposure}} {\cite{rubber_industry}} or present in
      tobacco smoke {\cite{smoking_cancer}} are known to increase cancer risk.
      Cellular metabolic processes also produce DNA-damaging products that
      induce cancer, such as reactive oxygen species {\cite{ros_cancer}}
      {\cite{ros_cancer_other}}. Several cellular DNA repair systems have
      emerged. DNA lesions can escape these repair mechanisms if the damage
      happens in an inaccessible region of the DNA or if the DNA repair system
      is defective {\cite{dna_repair}}. Also, the repair systems cannot cope
      with the rate of mutation if the frequency at which they occur becomes to
      important. These DNA lesions, if not repaired, then induce errors in the
      replication by DNA polymerases.

      The process of DNA replication is not free from errors. It has been
      estimated that DNA polymerase has error rates ranging from
      10\textsuperscript{-4} to 10\textsuperscript{-6}. This is followed by
      mismatch repair, which corrects 90-99\% of the replication errors,
      decreasing the overall error rate to 10\textsuperscript{-6} to
      10\textsuperscript{-8} {\cite{multiple_mutations}}. Additionally, several
      DNA polymerases exist, which differ in their error rates, which can be
      used interchangeabilly. DNA polymerase $\beta$ has a much worse error rate
      than DNA polymerase $\delta$ or $\epsilon$. There is evidence that that
      DNA $\beta$ is increased in some tumors {\cite{dna_pol}}, resulting in
      increased mutagenesis.

      Additionally, processes affecting chromosomal and microsatellite integrity
      instability contribute to genomic instability in cancer cells.
      \textbf{Chromosomal instability (CIN)} is the most common kind of
      instability in solid tumors {\cite{cin_crc}}. Chromosome missegregation
      plays a crucial role in cancer adaptation
      {\cite{chromosome_missegregation}}. Defects in proteins needed for
      chromosome segregation lead to chromosome missegregation. This leads to
      telomere dysfunction,  faulty sister chromatid cohesion, loss of
      heterozygosity (LOH), hypo-- or hyperactive spindle assembly checkpoint or
      defective centrosome duplication and aneuploidy. {\cite{cin_crc}}. About
      70\% of solid tumors are aneuploid {\cite{aneuploidy}}. Another
      chromosomal instability process has been described recently:
      chromothripsis happens when chromosomes are fragmented
      {\cite{chromothripsis_1}} {\cite{chromothripsis_2}}
      {\cite{chromothripsis_2}}. The cell tries to repair the chromosomes, but
      this process is far from being perfect, leading to massive chromosomal
      rearrangements. The question whether CIN is a cause or consequence of
      tumor development remains unanswered.

      \textbf{Microsatellite instability (MSI)} is a phenotype caused by
      inactivation or loss of DNA mismatch repair {\cite{msi}}. Microsatellites
      are short DNA segments with tandem repeats. Microsatellite elongation or
      shortening is a consequence of defective or inactive DNA mismatch repair
      (MMR), which corrects base replication errors. Seven enzymes contribute to
      the MMR system (MLH1, MLH3, MSH2, MSH3, MSH6, PMS1, PMS2)
      {\cite{cin_crc}}. Germline mutations in MMR genes cause the Lynch syndrome
      (hereditary nonpolypsos colorectal cancer) {\cite{lynch}}. Patients have
      an 80\% lifetime risk to develop colon cancer. Germline LOH of one allele
      with somatic inactivation on the other allele or double allelic
      inactivation by somatic mutations of these genes can cause MSI. The most
      common reason for MMR inactivation is through methylation of the promoter
      of the MLH1 gene {\cite{lynch_2}}. DNA polymerase has a higher error rate
      in repetitive regions. When MMR genes are inactivated or defective, the
      replication mistakes in microsatellites cannot be corrected: MSI is the
      consequence. In some cancers, MSI can occur despite functional MMR through
      frameshift mutations at microsatellites. MSI is often associated with
      cancers harboring mutations in TGF$\beta$RII, EGFR, PTEN, and BAX, which
      contain such simple repeats {\cite{micro}}.

      \textbf{Epigenetic changes}

      Blablabla

    \subsubsection{Driver and passenger mutations}

      Genomic instability in cancerous cells becomes a critical mechanism if it
      affects oncogenes or tumor suppressor genes, which have the potential to
      be causative tumor 'driver' mutations. These driver mutations are
      positively selected during cancer progression and confer a growth
      advantage to the cells harboring them. Many alterations found in cancer
      cells are passenger mutations, which occur subsequently or coincidentally
      to driver mutations. These mutations are defined to not contribute to the
      selective fitness of the cell, even though this conception has been
      challenged by stochastic tumor progression simulations
      {\cite{stochastic_cancer}}. Some studies have reported that cancer cells
      carry 40--80 somatic mutations, and only 5--15 of them are driver
      mutations {\cite{som_mut}}.

      Estimating the number of somatic driver and passenger mutations and the
      rate at which they occur is not well established {\cite{driver_passenger}.
      Two tumors, even though histologically indistinguishable, might present
      different subsets of mutations {\cite{driver_passenger}
      {\cite{intertumor}}. This observation has been defined as inter-tumor
      heterogeneity. Additionally, tumors present heterogeneity at the
      intra-tumor level {\cite{intratumor}. Subclones of the tumor might present
      different mutations.

      As mentioned, cancer progression is an evolutionary process. Chemotherapy
      creates a selective environment {\cite{selective_chemo}}. Initially,
      patients often respond to the therapy, but might then become resistant to
      the treatment. This is due to intra-tumor heterogeneity: a subclone of the
      tumor might have acquired a driver mutation that confers resistance to the
      treatment. Chemotherapy might kill a large part of the tumor cells, but
      actively selects for this resistant clone. Eventually, this clone will be
      the origin of relapses and another treatment option is lost.

      The identification of driver mutations has been a central aim of cancer
      research. Of the 20,000 protein coding genes, mutations in at least 350
      human genes are found recursively in cancer genomes and are believed to
      contribute to cancerogenesis {\cite{cancer_genome}}. Studies in mice have
      demonstrated that mutations in a total of 2,000 genes might contribute to
      cancer development and progression. According to their functional
      importance, these genes can be divided into two classes: tumor suppressor
      genes and oncogenes.

      \paragraph{Tumor suppressor} genes protect a cell from entering the path
      to cancer. They comprise genes encoding for cell adhesion proteins,  DNA
      repair proteins, proteins acting in apoptosis pathways, or cell cycle
      proteins. The action of these proteins inhibits metastasis, excessive cell
      survival or proliferation. Tumor suppressors mostly follow the two-hit
      hypothesis, which was first proposed by Knudson  for the retinoblastoma
      protein (pRb): to inactivate the tumor-protecting role of tumor
      suppressors, two genetic events, often LOH in  combination with silencing
      point mutations or silencing of both alleles by somatic events, are
      necessary to inactivate both alleles of the gene. Another possibility of
      tumor suppressor inactivation is methylation of the gene promoter.
      Compared to dominant oncogenes, tumor suppressor genes are often
      considered to be recessive. Alternatively, tumor progression can be
      influenced by functional haploinsufficiency of tumor suppressors.
      According to this conception, a disease state can emerge if a cell /
      organism has only one functional copy of a given gene and if it cannot
      produce enough of a gene product to establish a wild-type condition. APC,
      TP53 and the TGF-b pathway are amongst the most known tumor suppressors.

      Adenomatous Polyposis Coli (APC) is a protein, which has binding sites for
      microtubules, cytoskeletal regulator proteins and Wnt signaling proteins
      ($\beta$--catenin, axin). Wnt signaling regulates cell migration,
      polarity, differentiation, adhesion and apoptosis. In the canonical Wnt
      signaling pathway, a destruction complex, including APC, leads to
      $\beta$--catenin phosphorylation, followed by ubiquitination, marking it
      for degradation in the proteasome. Once Wnt binds to the N-terminus of an
      activated surface receptor of the Frizzled family and a
      co-receptor of the LRP5/6 family, the destruction complex is inhibited.
      Consequently, $\beta$--catenin is no longer marked for degradation and can
      translocate to the nucleus, where it acts on gene
      expression of target genes. Loss or dysfunction of APC leads to
      $\beta$--catenin accumulation in the nucleus even in the absence of  an
      extracellular stimulus.

      TP53 is one of the master guardians of the genome. In normal situations,
      p53, the protein encoded by TP53, is targeted for ubiquitination and
      degradation in the proteasome. In case of cellular stress, p53 is no
      longer ubiquitinated. p53 can then stop the cell cycle at the G1/S and
      G2/M transitions, induce DNA repair, and induce apoptosis if the damage
      cannot be repaired. TP53 thereby maintains genomic stability. The
      importance of TP53 as tumor suppressor gene becomes evident in the
      autosomal dominant Li--Fraumeni syndrome. People suffering from this
      disorder inherit only one functional copy of TP53 and are likely to
      develop cancer in early ages. One mechanism by which p53 acts on
      cell-cycle arrest is by activating expression of p21. p21 binds to the
      G1/S transition complex and inhibits its activity, leading to cell-cycle
      arrest. Inactivation or mutation of TP53 is a crucial step in many
      cancers.

      \paragraph{Oncogenes} comprise several GTPases, transcription factors,
      receptor tyrosine kinases and growth factors. Overexpressed or overactive
      versions of these proteins lead to increased mitogenic signals,
      causing increased cell growth or proliferation. Mutations in
      proto--oncogenes can cause a loss of regulation or overactive proteins.
      Gene duplications or other chromosomal alterations lead to increased
      protein synthesis. Other mechanisms of importance include
      post-transcriptional mechanisms as misregulation of protein expression or
      increase of mRNA / protein stability. Two important oncogenic pathways
      include the RAS--RAF--MAPK and PTEN--PI3K--AKT pathways.

      Signaling through the PI3K--AKT pathway leads to cell growth,
      proliferation and survival. The signaling cascade is initiated by
      integrins, cytokine receptors, T and B cell receptors, G--protein coupled
      receptors receptor tyrosine kinases, such as the Epithelial Growth Factor
      (EGFR). Ligand binding results in production of PIP3
      (phosphatidylinositol--(3,4,5)) by activation of PI3K
      (phosphoinositide--3--kinase). PIP3 is anchored in the cell membrane and
      acts as docking site for proteins containing PH domains (pleckstrin--
      homology), such as PDK1. PIP3-bound PDK1 partially activates Akt by
      phosphorylation of its Thr308. Full activation of Akt is enabled by
      phosphorylation pf PDK1 at Ser473 by mTORC2. Activated Akt then acts on a
      variety of proteins necessary for protein synthesis, glucose metabolism,
      cell survival / death and proliferation. The phosphatases PP2A and PHLPP
      can dephosphorylate and thereby inactivate Akt. Additionally, PTEN
      dephosphorylates PIP3 and indirectly also inactivates Akt. Dysregulation
      of the PI3K--AKT has been associated with several human diseases including
      neurological diseases, diabetes and cancer. In cancer, inactivation of
      PTEN and kinase activity activating mutations on PI3K and Akt are found
      recursively, leading to enhanced signaling, leading to inhibition of
      apoptosis and increased proliferation.

      In the RAS--RAF--MEK--ERK pathway, ligand binding on cell surface receptor
      tyrosine kinases activates the receptor. One of these receptors is the
      EGFR. EGFR is a protein of the tyrosine kinase receptor family. It is
      anchored in the cytoplasmic membrane and is composed of an
      intracytoplasmic tyrosine kinase domain, a short hydrophobic transmembrane
      domain and an extracellular ligand-binding domain. Ligand binding causes a
      conformational change of the receptor, which leads to homo-- or
      heterodimerization, followed by an auto-- and cross--phosphorylation of
      key tyrosine residues on its cytoplasmic domain. This forms docking sites
      for cytoplasmic proteins that contain phosphotyrosine-binding and Src
      homology 2 domains. GRB2 binds to Tyr1068 of EGFR through its SH2 domain
      and recruits SOS, a guanine nucleotide exchange factor. Grb2 and SOS then
      form a complex with the activated EGFR, which activates SOS. Activated SOS
      promotes recruition of Ras proteins to the activated EGFR. Through its GEF
      activity, SOS then induces GDP removal from Ras proteins, which can
      subsequently bind GTP and become active. Ras then recuits Raf proteins to
      the cell membrane and binds to their N-terminus. The activation of Raf,
      serine/threonine kinase proteins, is complex. In fact, Raf proteins are
      considered as gatekeepers of the RAS--RAF--MAPK pathway. In its inactive
      form, Raf is present in a 'closed' conformation, in which an
      autoinhibitory domain blocks the catalytic kinase domain. Recruitment to
      the cell membrane of Raf by Ras results in a conformational change, which
      disrupts the autoinhibitory interaction of Raf. Rafs then form homo-- or
      heterodimers, which leads to partial activation by allostery.
      Transphorylative events, with optional phosphorylation by other kinases,
      such as PAK1, then fully activates Raf. Activated Raf can now bind to
      MEKS, which are tyrosine/threonine kinases. MEKs then phosphorylate ERKSs,
      which are also serine/threonine kinase enzymes. ERKs then translocate to
      the cell nucleus, where they influence expression of target genes.
      RAS--RAF--MEK--ERK signaling promotes cell-cycle progression, cell
      differentiation, growth and survival.

  \subsection{Targeting the EGFR signaling pathway}

    \subsubsection{Molecular Profiling of Solid Tumors}

    \textbf{Lung cancer} is the most common cancer worldwide, both in terms of
    new cases (1.8 million) and deaths (1.6 million). Smoking is a widely
    accepted risk factor, as chemical carcinogens in tobacco smoke induce
    several genetic mutations. Classic diagnosis and treatment decisions have
    relied on histological analysis of the tumor. Lung cancer can be divided
    into two histological subtypes: small-cell lung cancer (SCLL) and non-small
    cell lung cancer (NSCLC). Over the last decade, it has become clear that
    these subtypes can be classified into  additional classes by the mutational
    status of recurrent driver mutations in genes that are frequently mutated in
    this type of cancer. Mutations in several oncogenes can be found at each
    stage of NSCLC and in all histological types, e.g. large cell carcinoma,
    adenocarcinoma, squamous cell  carcinoma (SCC), in smokers and never-smokers
    as well.

    A combination of oncogenic triggers cause cells of the normal bronchial
    epithelium to proliferate, giving rise to meta--, hyper-- and dysplastic
    epithelial lesions. Genomic events in early stages of lung cancer giving
    rise to atypical adenomatous hperplasia include LOH of 3p (80\%),
    p16\textsuperscript{INK4a} (70\%) or RB inactivation (15\%), as well as
    mutations in KRAS (15--25\%) or in $\beta$--catenin (10\%). TP53
    inactivation (50\%) and LOH at 13q are believed to favor progression into
    the primary adenocarcinoma stage. After that stage, major chromosomic
    instability is often detected, giving rise to metastatic adenocarcinoma.
    These chromosomic events include LOH of 2q (70\%), 9p (80\%), 18q (85\%),
    and 22q (75\%). Additionally, the oncogene c-myc is amplified in 10\%.

    Frequent  mutations in NSCLC, which are of potential interest in a targeted
    anti-tumor chemotherapy, affect EGFR (10--35\%), KRAS (15--25\%), PTEN
    (4--8\%), HER2 (2--4\%), DDR2 (4\%), PIK3CA (1--3\%), BRAF (1--3\%), AKT1
    (1\%), MEK (1\%) and NRAS (1\%). Additionally, rearrangement of ALK
    (3--7\%), RET (1\%) and ROS1 (1\%) and amplifications of FGFR1 (20\%) and
    MET (2--4\%) are found recursively. These mutations are rarely observed
    together in the same tumor.

    \textbf{Melanoma} develops from the malignant transformation of melanocytes
    in the basal epidermal layer of the skin. Melanoma incidence has exploded
    over the last four decades, with a 15-fold increase in the United States.
    Both genetic predisposition and environmental factors influence the risk of
    getting melanoma. Skin cancer often affects fair-skinned individuals.
    Exposure to UV light, immunosuppression and multiple nevi are risk factors.
    UV radiation causes cyclobutane pyrimidine dimers (CPDs). By joining
    adjacent pyrimidine bases, T--T, C--C or C--T dimers (UV fingerprints) are
    formed, leading to direct DNA damage. People diagnosed with rare genetic
    disorders like xeroderma pigmentosum are at great risk. Traditionally,
    melanoma has been classified based on histological and pathological
    properties, such as the thickness of the tumor, ulceration or the anatomic
    location of the tumor.



    Several mutations in oncogenes are recurrently found in melanoma, but are
    rarely found together in the same tumor. The occurrence of the different
    mutations differs by the anatomic location of the tumor, e.g. whether the
    specific body part is chronically exposed to the sun. Mutations frequently
    found in melanoma occur on BRAF (37--50\%), NRAS (13--25\%), MEK (6--7\%),
    NF1 (11.9\%), CTNNB1 (2-4\%), GNAQ (1.3\%) and GNA11 (1.2\%).

    \textbf{Colorectal cancer (CRC)} 1.4 million cases are detected yearly with
    694,000 deaths. CRC is one of the best studied cancers. The development of
    colorectal adenocarcinomas occurs over many years. Caused by the acquisition
    and accumulation of driver mutations, a normal colorectal epithelium can
    progress to adenoma, which develops into carcinoma, which can eventually
    mestastasize.



    Mutations recursively detected in CRC occur on KRAS (36--40\%), SMAD4
    (10--35\%), PIK3CA (10--30\%), BRAF (8--15\%), PTEN (5--14\%), NRAS
    (1--6\%), and  AKT1 (1--6\%).

    The mentioned molecular progression profiles are likely to be an
    oversimplification. Due to tumor heterogeneity, these alterations do not
    always have to be observed in the tumor, and the chronological appearance
    of these alterations may vary from one tumor to another.

    \subsubsection{Biological Role of EGFR in Solid Tumors}

      In normal cells, the tighly regulated EGFR signaling pathway drives
      cell-cycle progression, affects differentiation and migration and  acts
      as a survival signal. EGFR ligand binding leads to an activation of
      several signaling cascades, such as the PI3K, RAS--RAF--MEK--ERK, JAK--STAT
      and NFkB pathways. The EGFR pathway is long known to be dysregulated
      in most solid tumors. EGFR levels have been demonstrated to be higher in
      CRC samples than in surrounding tissues. Also, more of EGFR's ligands EGF
      and TGF$\alpha$ are found in these locations. Increased EGFR mRNA levels
      in in vitro culture of human CRC cells  has also linked EGFR
      overexpression to tumor progression. EGFR overexpression has been
      associated with poor prognosis, as this leads to a more aggressive
      progression.

      It was suggested that EGFR overexpression results from epigenetic
      alterations, leading to enhanced gene expression. Gene amplification and
      oncogenic viruses may also cause EGFR overexpression.

      http://annonc.oxfordjournals.org/content/16/1/102.full
      http://www.nature.com/onc/journal/v28/n1s/full/onc2009199a.html

      \paragraph{EGFR signaling as a survival signal}

        Cells communicate with their environment. Without extracellular signals,
        a cell undergoes apoptosis. Signaling pathways induced by cell--matrix
        interactions, cell-cell interactions, and soluble survival factors act
        on a variety of genes and proteins.

        Loss of matrix attachment leads to cell growth arrest and even cell
        death in normal epithelial cells, a process called anoikis. The
        cell--matrix interaction provides important spatial informations to the
        cell and acts as a safeguard against inappropriate proliferation and
        migration.

        Activation of the EGFR signaling pathway allows protection of normal
        epithelial cells against anoikis in the suspended state. EGFR blockade
        sensitizes normal epithelial cells to apoptosis. This effect is much
        more pronounced in the suspended state than in the attached state. The
        redundancy of cell survival signals, makes normal epithelial cells
        relatively resistant to EGFR-blockade in their normal microenvironment.
        MAPK activation in the RAS--RAF--MAPK--ERK pathway is required for high
        expression of Bcl--XL, an anti-apoptotic protein of the Bcl--2 family.
        Bcl--2 can be either pro-- or anti--apoptotic and regulate liberation of
        cytochrome c from the mitochondria, which is essential in the apoptotic
        caspase pathway. Expression of pro--apoptotic Bcl--2 proteins (Bax, Bad,
        Bak) are not influenced by EGFR signaling. Additionally, EGFR signaling
        leads to post-transcriptional phosphorylation on the pro-apoptotic Bad
        protein, which is thereby functionally inactivated.

        Tumor cells are often in transit or at sites with inadequate matrix
        composition. They are often provided with inadequate or missing
        cell--cell or cell--matrix interactions. They are thereby more dependent
        on survival signals propagated by soluble mediators, such as EGF or
        TGF$\alpha$. This can counterbalanced by an upregulation of cell surface
        receptors that activate anti--apoptotic pathways. Consequently, tumor
        cells are relatively sensitive against blockade of EGFR (or other cell
        surface receptors. Due to inappropriate interactions, these cells
        heavily depend on survival signals propagated by soluble mediators,
        such as EGF or TGF$\alpha$.

    \subsubsection{EGFR-targeted drugs}

      The observations that EGFR is recursively upregulated in many cancers
      and that EGFR is such an important mediator of not only cell-cycle
      progression and cell growth, but also cell survival, has lead to the
      development of agents that block this pathway. Pharmacologically,
      these agents can be classed by their mode of action: EGFR-targeted
      monoclonal antibodies and EGFR-specific tyrosine kinase inhibitors.
      Additionally, several proteins acting downstream of EGFR are recursively
      found to be mutated. This lead to the development of other targeted
      drugs, such as BRAF-- or MEK--inhibitors. Table XXX shows a selection
      of FDA-approved cancer drugs that target components of the EGFR signaling
      pathway.

      \begin{table}[!htbp]
          \caption[Targeted Cancer Agents]{FDA-approved cancer drugs for solid tumor treatment that target the EGFR pathway}
          \centering
          \begin{tabular}{ |p{4cm}|p{3.7cm}|p{6.3cm}|}
          \hline
          Agent & Target(s) & FDA-approved indication(s) \\ \hline \hline
          Afatinib (Gilotrif) & EGFR, HER2 (ERBB2/neu) & NSCLC (with EGFR del19 or L858R) \\
          Cetuximab (Erbitux) & EGFR & Colorectal cancer (KRAS WT) \\
          Cobimetinib (Cotellic) & MEK & Melanoma (with BRAF V600E or V600K \\
          Dabrafenib (Tafinlar) & BRAF & Melanoma (with BRAF V600 mutation) \\
          Erlotinib (Tarceva) & EGFR & NSCLC \\
          Gefitinib (Iressa) & EGFR & NSCLC (with EGFR del19 or L858R) \\
          Necitumumab (Portrazza) & EGFR & Squamous NSCLC \\
          Osimertinib (Tagrisso) & EGFR & NSCLC (with EGFR T790M) \\
          Panitumumab (Vectibix) & EGFR & Colorectal cancer (KRAS WT) \\
          Trametinib (Mekinist) & MEK & Melanoma (with BRAF V600) \\
          Vemurafenib (Zelboraf) & BRAF & Melanoma (with BRAF V600) \\
          \hline
        \end{tabular}
      \end{table}

      Anti--EGFR monoclonal antibodies bind to the extracellular domain of EGFR
      in its inactive state. They thereby compete for receptor binding by
      occluding the ligand-binding domain. They thereby inhibit ligand-induced
      EGFR tyrosine kinase activation. The two most anti--EGFR monoclonal
      antibodies used are cetuximab and panitumumab.

      Cetuximab binds to EGFR with a higher affinity than the natural ligands
      EGF or TGF$\alpha$. Cetuximab binding induces internalization of EGFR and
      subsequent degradation. Cetuximab also binds to EGFRvIII, a constantly
      active version of EGFR. Cetuximab thereby induces apoptosis by increasing
      expression of pro-apoptotic proteins (e.g. Bax and caspase-3, caspase-8
      and caspase-9) or by inactivation of anti-apoptotic proteins (e.g. Bcl-2)
      inducing decreased expression or phosphorylation. Binding of cetuximab to
      EGFR inhibits the progression of the cell cycle at the G0/G1 boundary.
      Cetuximab has also been reported to inhibit the production of
      pro-angiogenic factors such as vascular endothelial growth factor,
      interleukin-8 and the basic fibroblast growth factor; inhibition of these
      factors is associated with a decrease in new blood vessel formation and
      the development of distant metastases in orthotopic cancer models.

      www.ncbi.nlm.nih.gov/pmc/articles/PMC2759052/

      EGFR-specific tyrosine kinase inhibitors comprise three classes that
      include first generation reversible  EGFR-inhibitors (gefitinib,
      erlotinib), second generation irreversible inhibitors (afatinib,
      dacomitinib, neratinib) and third generation mutant-selective inhibitors
      (brigatinib, osimertinib, rociletinib). Third class agents have a better
      sensitivity against mutated than wild-type EGFR and have been designed to
      further decrease treatment-associated side effects. Tyrosine
      kinase inhibitors are low molecular weight molecules that are mainly
      derived from quinazoline. These compounds bind to EGFR and block
      ligand-induced receptor phosphorylation by competing for the ATP--
      binding site.

      Gefitinib inhibits the growth of a range of human cancer cells in vitro
      and in vivo, and there is evidence that the inhibitor acts by inducing
      cell-cycle arrest and/or apoptosis. Erlotinib inhibits proliferation of
      DiFi human colon tumor cells at submicromolar concentrations in cell
      culture, blocks cell-cycle progression at the G1 phase and triggers
      apoptosis. At doses of 100 mg/kg, erlotinib completely prevents
      EGF-induced phosphorylation of EGFR in human HN5 tumor xenografts in
      athymic mice. Combination chemotherapy of erlotinib with cisplatin
      produces a significant response greater than that of cisplatin alone, with
      no detected effects on body weight or lethal toxicity.

      Cetuximab, ABX-EGF, gefitinib, and erlotinib have all been evaluated in
      clinical trials both as single agents and in combination with conventional
      chemotherapy or radiation therapy. Because EGFR inhibition and
      conventional anticancer therapy act via different cytotoxic mechanisms,
      combination therapy offers the potential advantages of additive or
      synergistic activity without overlapping toxicity profiles.

      http://erc.endocrinology-journals.org/content/11/4/689.long

    \subsubsection{Predictive markers}

      Wild-type status of EGFR, KRAS, NRAS and BRAF has been associated with
      increased sensitivity to anti-EGFR antibodies.

      \paragraph{EGFR} is a strong predictive biomarker for the success of the
      administration of EGFR-specific tyrosine kinase inhibitors. EGFR
      activating mutations are observed in 10--35\% of NSCLC. 90\% of these
      mutations are exon 19 deletions and exon 21 L585R (c.2573T>G) point
      mutations. Amongst EGFR-mutated NSCLC, EGFR L858R occurs with a frequency
      of 43\%.

      \paragraph{KRAS} mutations in CRC are found with a frequency of 36--40\%.
      Amongst these, the G12C variant is the most common (7.9\%). Critical
      mutations in the KRAS gene include variants in codons 12, 13 and 61. These
      mutations lock KRAS in its GTP-bound state, resulting in a constantly
      active protein. This then leads to a constantly active signal
      transduction. Blocking EGFR in that case is useless, as KRAS acts
      downstream of EGFR. Several KRAS point mutations in codons 12, 13 and 61
      have been shown to confer reduced sensitivity to EGFR-targeted monoclonal
      antibodies in CRC. The situation is similar in the case of the
      KRAS-isoform NRAS.

      \paragraph{BRAF} mutations are very common in melanoma (37--50\%). Amongst
      BRAF-mutated melanomas, the V600E variant is found in 80--90\% cases. This
      variant occurs in the activation segment of the BRAF kinase domain and
      results in increased kinase activity. BRAF V600E mutations have been
      associated with increased sensitivity to BRAF and MEK inhibitors.

  \subsection{Tumor DNA Sequencing}

    Cancer sequencing using next-generation sequencing (NGS) methods provides
    more information in less time compared to traditional single-gene and
    array-based approaches. With NGS, researchers can perform whole-genome
    studies, targeted gene profiling, tumor-normal comparisons, and more. NGS
    also offers the sensitivity to detect rare somatic variants, tumor
    subclones, and circulating DNA fragments.

    the field of cancer genomics has been impacted most profoundly by the
    application of next-generation sequencing technology, which has
    tremendously accelerated the pace of discovery while dramatically reducing
    the cost of data production.

    Many biological discoveries about cancer have been the product of a reductionist
    approach, which focuses on modeling phenomena with as few major actors and
    interactions as possible [1, 2]. This reductionist thinking led the initial
    theories on carcinogenesis to be centered on how many “hits” or genetic
    mutations were necessary for a tumor to develop. It was assumed that each type
    of cancer would progress through a similar, if not identical, process of genetic
    hits.

    However, most cancers are genetically complex, and are better defined by the
    activation of signaling pathways rather than a defined set of mutations. The
    success of the Human Genome Project inspired similar projects looking at the
    genome in various cancers [4]. That success, along with the increased
    affordability and reliability of sequencing [5], has led to the integration of
    genome science into clinical practice.

    http://www.ncbi.nlm.nih.gov/pmc/articles/PMC4276967/

    http://www.sciencedirect.com/science/article/pii/S1574789113000781

    http://journal.frontiersin.org/article/10.3389/fgene.2015.00215/full

    https://www.aslme.org/media/downloadable/files/links/0/3/03.SUPP_Deverka.pdf

    https://genomemedicine.biomedcentral.com/articles/10.1186/s13073-015-0203-x

    http://www.ncbi.nlm.nih.gov/pmc/articles/PMC3219767/

    https://www.thermofisher.com/lu/en/home/life-science/cancer-research/cancer-genomics/targeted-sequencing-cancer-mutation-detection/benefits-targeted-ngs-cancer-research.html

    http://www.cell.com/cell-systems/fulltext/S2405-4712%2815%2900113-1

    http://www.ncbi.nlm.nih.gov/pmc/articles/PMC3599179/

    http://www.comprehensivegenomicprofiling.com/

    \subsubsection{Targeted NGS}

      profiling of mutational status of some genes of interest
      point mutations and small insertions & deletions
      guide targeted cancer therapy

      \paragraph{Target enrichment methods}

    \subsubsection{Practical implications in the laboratory}

    FFPE (quantity, quality, C>T)
    Tumor:normal
    Tumor heterogeneity
    Choice of instrument & library preparation
    Bioinformatic pipeline + expertise to interprete data

  \subsection{Aims of the Thesis}
