\section{Introduction}

  Cancer represents a huge burden for health care systems worldwide and one of
  the leading death causes. In 2012, there were an estimated 14.1 million new
  cancer cases with estimated  8.2 million cancer deaths
  {\cite{cancer_stats_worldwide:2012}}. Lung cancer is the most common cancer,
  both in terms of new cases (1.8 million) and deaths (1.6 million). Breast
  cancer is the second most common cancer (1.7 million cases) but only ranks 5th
  as cause of death (522,000 deaths). Colorectal cancer (1.4 million cases;
  694,000 deaths), prostate cancer (1.1 million cases; 307,000 deaths), stomach
  cancer (951,000 cases; 723,000 deaths) and liver cancer (782,000 cases;
  723,000 deaths) are following.

  Scientific discoveries in the last decade have had an enormous impact on our
  understanding of the underlying causes of cancer. The development of omics
  techniques, in combination with enhanced computational power, has lead to an
  explosion of biological data. It has become clear that cancer is an incredibly
  complex malignancy. The research community is trying to interprete this vast
  amount of data with the goal to get a deeper understanding of cancer and to
  cure it eventually. In recent years, several drugs have been approved that
  target proteins needed for cancer development, proliferation or metastasis.
  Molecular testing is employed to check whether these targeted drugs would be
  of benefit. In that regard, Next-Generation Sequencing (NGS) is an interesting
  method to gain deep insights into the genetic information of a tumor and to
  guide personalized therapy.

  \subsection{Targeting Cancer}

    Virtually all cancers tend to accumulate mutations during their progression.
    The genetic diversity caused by this instability, the cardinal feature of
    cancer, in combination with several environmental factors, such as
    inflammation, enables the hallmarks of cancer (Hallmarks of Cancer: The Next
    Generation). These include replicative immortality, cell death resistance,
    ongoing proliferative signaling, invasion and metastasis, growth suppressor
    evasion, inducement of angiogenesis, energy metabolism reprogramming and
    immune destruction evasion.

    Cancer development is considered to be a state of imbalance between the
    occurrence of mutations and cell-cycle control mechanisms. In many cancers,
    several pathways contribute to the abnormal cell growth: chromosomal
    instability (CIN), microsatellite instability (MSI) and the CpG island
    methylator phenotype (CIMP).

    Large-scale studies have demonstrated that cancers are highly heterogeneous.
    This heterogeneity has been observed both at the inter- and intra-level. Two
    tumors of similar phenotype often comprise a different subset of mutations
    that may even have a low overlap. Several clonal subpopulations within the
    same tumor contribute to the intra-tumor heterogeneity. This is a
    considerable problem in the clinics, as some subpopulations of a cancer may
    become resistant to the  treatment and may be the source of relapses.

    Many alterations found in cancer cells are passenger mutations, e.g. do not
    contribute to the selective fitness of the cell. Driver mutations, often
    happening on oncogenes or tumor suppressor genes, promote the cell's
    fitness. This concept recognizes Darwinian evolution principles. The
    heterogeneous  population of cancer cells harbors cells with different
    random somatic and non-deleterious mutations and exhibits different
    perturbations. Cells with the best fitness, e.g. with the highest
    proliferative potential  and the lowest death rate, are then selected
    through natural selection principles. These cells will outlast less fit
    cells. This results in  sequential waves of clonal expansion, leading to
    different subclones within the same tumor that differ in their
    proliferative, migrative and invasive potential. The hypothesis that
    passenger mutations that occur subsequently or coincidentally to driver
    mutations do not influence the cell's fitness at all has been
    challenged by stochastic tumor evolution simulations (citat). Since then,
    it has been proposed that, even though the individual effect may be
    small, the cooperation of multiple accumulated small-scale passenger
    mutations plays a present role in cancer development and progression.

    Genomic instability in cancerous cells becomes a critical mechanism if it
    affects oncogenes or tumor suppressor genes. Tumor suppressor genes protect
    a cell from entering the path to cancer. They comprise
    genes encoding for cell adhesion proteins,  DNA repair proteins, proteins
    acting in apoptosis pathways, or or cell cycle proteins. The action of these
    proteins inhibits metastasis, excessive cell survival or proliferation. Tumor
    suppressors mostly follow the two-hit hypothesis, which was first proposed
    by Knudson  for the retinoblastoma protein (pRb): to inactivate the
    tumor-protecting role of tumor suppressors, two genetic events,
    often LOH in  combination with silencing point mutations, are necessary to
    inactivate both alleles of the gene. Another possibility of tumor
    suppressor inactivation is methylation of the gene promoter. Compared to
    dominant oncogenes, tumor suppressor genes are often considered to be
    recessive. Alternatively, tumor progression can be influenced by functional
    haploinsufficiency. According to this conception, a disease state can emerge
    if a cell / organism has only one functional  copy of a given gene and if it
    cannot  produce enough of a gene product to establish a wild-type condition.
    Oncogenes comprise several GTPases, transcription factors, receptor tyrosine
    kinases and growth factors. Overexpressed or overactive versions of these
    proteins often lead to increased mitogenic signals, causing increased
    cell growth or proliferation. Mutations in proto--oncogenes can cause a
    loss of regulation or overactive proteins. Gene duplications or other
    chromosomal alterations lead to increased protein synthesis. Other
    mechanisms of importance include post-transcriptional mechanisms as
    misregulation of protein expression or increase of mRNA / protein stability.

    The EGFR signaling pathway is an oncogenic pathway frequently altered in
    many cancers. EGFR signaling promotes metastasis, invasion, cell survival
    and proliferation by activating the RAS-RAF-MAPK and PI3K-Akt pathways. EGFR
    is part of the family of receptor tyrosine kinases. This transmembrane
    protein is composed of an intracytoplasmic tyrosine kinase domain, a short
    hydrophobic transmembrane region and an extracellular ligand-binding domain.
    Ligand (EGF, TGFα) binding causes homo-- or heterodimerization, followed by
    an auto-- and cross--phosphorylation of key tyrosine residues on its
    cytoplasmic domain. This forms docking sites for cytoplasmic proteins that
    contain phosphotyrosine-binding and Src homology 2 domains.Activation of
    PTEN/PI3K/AKT leads to cell growth, proliferation and survival, while
    RAS-RAF-MAPK induces cell survival and cell cycle progression and
    proliferation. In the RAS-Raf-MAPK pathway, Grb2 and Sos, two adaptor
    proteins, form a complex with the activated EGFR. The resulting
    conformational change of Sos recruits Ras-GDP, which in turn becomes
    activated to form Ras-GTP. Ras-GTP activates Raf, which, in intermediate
    steps, phosphorylates a MAPK (mitogen-activated protein kinase). Activated
    MAPKs are then imported from the cytoplasm into the nucleus where they act
    on target genes. +++ PTEN/PI3K/AKT

    Many cancers have been shown to be dependent on EGFR signaling: they
    overexpress EGFR or to harbor activating mutations in EGFR or downstream
    proteins, both leading to increased mitogenic signals. Targeting the EGFR
    signaling pathway is an attractive target blabla and has been of benefit in
    solid tumors, e.g. melanoma, CRC and NSCLC. Advantages have been made better
    survival rates. shutting egfr down -> apoptosis. but 2 problems: resistances &
    not all mutations are actionable. therefore: molecular testing. the more
    comprehensible, the better. classical approaches only target some hotspot
    regions. NGS has the potential to give really deep insights

    \begin{table}[!htbp]
        \caption[Occurrence of mutations]{EGFR signaling pathway components affected in colorectal cancer, melanoma and non-small cell lung carcinoma}
        \centering
        \begin{tabular}{ |p{4cm}|p{3.7cm}|p{6.3cm}|}
        \hline
        Gene & Frequency in CRC (\%) & Frequency in Melanoma & Frequency in NSCLC \\ \hline \hline
        EGFR & NA & NA & 10--35 \\
        KRAS & 36--40 & 2 & 15--25 \\ (http://www.nature.com/bjc/journal/v112/n2/full/bjc2014476a.html)
        NRAS & 1--6 & 13--25 & 1 \\
        BRAF & 8--15 & 37--50 & 1--3 \\
        PTEN & 5--14 & NA & 4--8 \\
        PIK3CA & 10--30 & & 1--3 \\
        \hline
      \end{tabular}
    \end{table}


    \begin{table}[!htbp]
        \caption[Targeted Cancer Agents]{FDA-approved cancer drugs for solid tumor treatment that target the EGFR pathway}
        \centering
        \begin{tabular}{ |p{4cm}|p{3.7cm}|p{6.3cm}|}
        \hline
        Agent & Target(s) & FDA-approved indication(s) \\ \hline \hline
        Afatinib (Gilotrif) & EGFR (HER1/ERBB1), HER2 (ERBB2/neu) & NSCLC (with EGFR del19 or L858R) \\
        Cetuximab (Erbitux) & EGFR (HER1/ERBB1) & Colorectal cancer (KRAS WT) \\
        Cobimetinib (Cotellic) & MEK & Melanoma (with BRAF V600E or V600K \\
        Dabrafenib (Tafinlar) & BRAF & Melanoma (with BRAF V600 mutation) \\
        Erlotinib (Tarceva) & EGFR (HER1/ERBB1) & NSCLC \\
        Gefitinib (Iressa) & EGFR (HER1/ERBB1) & NSCLC (with EGFR del19 or L858R) \\
        Necitumumab (Portrazza) & EGFR (HER1/ERBB1) & Squamous NSCLC \\
        Osimertinib (Tagrisso) & EGFR & NSCLC (with EGFR T790M) \\
        Panitumumab (Vectibix) & EGFR (HER1/ERBB1) & Colorectal cancer (KRAS WT) \\
        Trametinib (Mekinist) & MEK & Melanoma (with BRAF V600) \\
        Vemurafenib (Zelboraf) & BRAF & Melanoma (with BRAF V600) \\
        \hline
      \end{tabular}
    \end{table}

  \subsection{Targeted Sequencing}

    \subsubsection{Target Enrichment Methods}

    \subsubsection{Illumina Sequencing Chemistry}

  \subsection{NGS Data Analysis}

    \subsubsection{GATK Best Practices}

  \subsection{Practical Implications in the Laboratory}

  \subsection{Aims of the Thesis}
