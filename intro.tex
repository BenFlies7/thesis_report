\section{Introduction}

  Cancer represents a huge burden for health care systems worldwide and one of
  the leading death causes. In 2012, there were an estimated 14.1 million new
  cancer cases with estimated  8.2 million cancer deaths
  {\cite{cancer_stats_worldwide:2012}}. Lung cancer is the most common cancer,
  both in terms of new cases (1.8 million) and deaths (1.6 million). Breast
  cancer is the second most common cancer (1.7 million cases) but only ranks 5th
  as cause of death (522,000 deaths). Colorectal cancer (1.4 million cases;
  694,000 deaths), prostate cancer (1.1 million cases; 307,000 deaths), stomach
  cancer (951,000 cases; 723,000 deaths) and liver cancer (782,000 cases;
  723,000 deaths) are following.

  Scientific discoveries in the last decade have had an enormous impact on our
  understanding of the underlying causes of cancer. The development of omics
  techniques, in combination with enhanced computational power, has lead to an
  explosion of biological data. It has become clear that cancer is an incredibly
  complex malignancy. The research community is trying to interprete this vast
  amount of data with the goal to get a deeper understanding of cancer and to
  cure it eventually. In recent years, several drugs have been approved that
  target proteins needed for cancer development, proliferation or metastasis.
  Molecular testing is employed to check whether these targeted drugs would be
  of benefit. In that regard, Next-Generation Sequencing (NGS) is an interesting
  method to gain deep insights into the genetic information of a tumor and to
  guide personalized therapy.

  \subsection{Targeting Cancer}

    It is widely accepted that cancer is a disease caused by genomic instability.
    The genetic diversity caused by this instability, in combination with several
    environmental factors, such as inflammation, enables the hallmarks of cancer
    (Hallmarks of Cancer: The Next Generation). These include replicative
    immortality, cell death resistance, ongoing proliferative signaling,
    invasion and metastasis, growth suppressor evasion, inducement of
    angiogenesis, energy metabolism reprogramming and immune destruction evasion.

    Virtually all cancers tend to accumulate mutations during their progression.
    driver & passenger mutations
    
    Cancer heterogeneity

    Genetic and genomic alterations (chromosome instability etc)

    Then oncogenetic pathways

    many solid tumors are dependent on EGFR

    EGFR pathway as target for drugs

    maybe a little about resistances

  \subsection{Targeted Sequencing}

    \subsubsection{Target Enrichment Methods}

    \subsubsection{Illumina Sequencing Chemistry}

  \subsection{NGS Data Analysis}

    \subsubsection{GATK Best Practices}

  \subsection{Practical Implications in the Laboratory}

  \subsection{Aims of the Thesis}
