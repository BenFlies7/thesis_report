\section{Introduction}

  Cancer represents a huge burden for health care systems worldwide and is one
  of the leading death causes. Scientific discoveries in the last decade have
  had an enormous impact on our understanding of the underlying causes of
  cancer. The development of omics techniques, in combination with advanced
  computational power, has lead to an explosion of biological data. It has
  become clear that cancer is an incredibly complex malignancy, which is
  affected by genetic, environmental and behavioural factors. The research
  community is trying to interprete this vast amount of data with the goal to
  get a deeper understanding of cancer and to cure it eventually. In recent
  years, several drugs have been approved, which target proteins needed for
  cancer development, proliferation or metastasis. Molecular testing is employed
  to check whether these targeted drugs would be of benefit. In that regard,
  Next-Generation Sequencing (NGS) is an interesting method to gain deep
  insights into the genetic information of a tumor and to guide personalized
  therapy.

  \subsection{The cancer genome}

    DNA undergoes continuous damage. In cancer cells, the equilibrium
    between DNA damage and repair systems is dysbalanced
    {\cite{dna_repair_epidemioloy}}. Genetic and epigenetic alterations, in
    combination with several environmental factors, such as inflammation, enable
    the hallmarks of cancer {\cite{cancer_hallmarks}}. These include replicative
    immortality, resistance to cell death, sustained proliferative signaling,
    invasion and metastasis, growth suppressor evasion, inducement of
    angiogenesis, energy metabolism reprogramming and evasion of immune
    destruction.

    It is widely accepted that tumors accumulate somatic mutations during their
    progression in malignancy {\cite{accumulation_rates}}
    {\cite{mutations_counting}}. DNA can be damaged by endogenous and
    environmental agents {\cite{multiple_mutations}}. Carcinogenic substances
    produced by industry {\cite{occupational_exposure}} {\cite{rubber_industry}}
    or present in tobacco smoke {\cite{smoking_cancer}} are known to increase
    cancer risk. Cellular metabolic processes also produce DNA-damaging products
    that induce cancer, such as reactive oxygen species {\cite{ros_cancer}}
    {\cite{ros_cancer_other}}. DNA lesions can escape DNA repair mechanisms if
    the damage happens in an inaccessible region of the DNA or if the DNA repair
    system is defective {\cite{dna_repair}}. Also, the process of DNA
    replication has an intrinsic error rate. It has been estimated that DNA
    replication by DNA plymerase followed by mismatch repair has an overall
    error rate to 10\textsuperscript{-6} to 10\textsuperscript{-8}
    {\cite{multiple_mutations}}. There is evidence that DNA $\beta$, a DNA
    polymerase with a higher error rate than DNA polymerase $\delta$ or
    $\epsilon$ is increased in some tumors {\cite{dna_pol}}, resulting in
    increased mutagenesis.

    Processes affecting chromosomal and microsatellite integrity instability
    contribute to genomic instability in cancer cells. \textbf{Microsatellite
    instability (MSI)} is caused by inactivation or loss of DNA mismatch repair
    {\cite{msi}}. Microsatellite elongation or shortening is a consequence of
    defective or inactive DNA mismatch repair (MMR), which corrects base
    replication errors {\cite{cin_crc}}. DNA polymerase has a higher error rate
    in repetitive DNA sequences. When MMR genes are inactivated or defective,
    the replication mistakes in microsatellites cannot be corrected: MSI is the
    consequence. In some cancers, MSI can occur despite functional MMR through
    frameshift mutations at microsatellites. MSI is often associated with
    cancers harboring mutations in TGF$\beta$RII, EGFR, PTEN, and BAX, as they
    contain such microsatellites {\cite{micro}}. Microsatellite and mismatch
    errors by DNA polymerase usually result in insertions or deletions and in
    point mutations, which can be silent, missense or nonsense. These mutations
    occur at the nucleotide level and  might affect the protein structure,
    leading to defective or overactive  enzymes.

    \textbf{Chromosomal instability (CIN)} is a common observation in solid
    tumors, especially in colorectal cancer {\cite{cin_crc}}. Chromosome
    missegregation plays a crucial role in cancer adaptation
    {\cite{chromosome_missegregation}}. Defects in proteins needed for
    chromosome segregation lead to chromosome missegregation. This leads to
    telomere dysfunction, faulty sister chromatid cohesion, loss of
    heterozygosity (LOH), hypo-- or hyperactive spindle assembly checkpoint or
    defective centrosome duplication and aneuploidy. {\cite{cin_crc}}. About
    70\% of solid tumors are aneuploid {\cite{aneuploidy}}. Another chromosomal
    instability process has been described recently: chromothripsis happens when
    chromosomes are fragmented {\cite{chromothripsis_1}}
    {\cite{chromothripsis_2}} {\cite{chromothripsis_2}}. The cell tries to
    repair the chromosomes, but this process is far from being perfect, leading
    to massive chromosomal rearrangements. The question whether CIN is a cause
    or consequence of tumor development remains unanswered.

    \textbf{Epigenetic changes} do not involve changes in the coding sequence,
    but affect gene expression. Gene expression can be influenced by histone
    modifications, dysregulation of DNA--binding of transcription factors or
    altered CpG island methylation. In normal cells, CpG islands in gene
    promoters are usually unmethylated. This is associated with active
    transcription. Other CpG islands across the genome are usually methylated.
    In cancer cells, this situation is often inverted. CpG islands in tumor
    suppressor promoters are found to be hypermethylated in many tumors. Their
    gene expression is thus drastically decreased. CpG methylation of tumor
    suppressor genes is found in 35--40\%  of colorectal cancers.

    \subsubsection{Driver and passenger mutations}

      Cancer progression is a process that recognizes basic Darwinian evolution
      principles {\cite{clonal_evolution}} {\cite{darwinian_models}}
      {\cite{war_zone}} {\cite{cancer_models}}. The population of cancer cells
      harbors heritable genetic variation. These mutations may be of germline
      origin or may occur through somatic processes. If the occurring mutations
      are non--deleterious, they can be passed on to the next generation of
      cells. The second process, which has to take place in Darwinian evolution,
      is natural selection. Each cell exhibits a unique combination of genetic
      and environmental perturbations. Cells are in competition for a variety of
      resources in their microenvironment, which include space, oxygen and
      nutrients. Eventually, cells with the best fitness, e.g. with the highest
      proliferative potential and the lowest death rate, are then selected
      through natural selection principles and will outlast less fit
      cells. Additionally, these cells will continue to accumulate new
      mutations. This results in sequential waves of clonal expansion
      {\cite{clonal_evolution}}.

      Genomic instability in cancerous cells becomes a critical mechanism if it
      affects oncogenes or tumor suppressor genes, which have the potential to
      be causative tumor 'driver' mutations. The identification of driver
      mutations has been a central aim of cancer research. Mutations in at least
      350 human genes are found recursively in cancer genomes and are believed
      to contribute to cancerogenesis {\cite{cancer_genome}}. These driver
      mutations are positively selected during cancer progression and confer a
      selective advantage to the cells harboring them. Many alterations found in
      cancer cells are passenger mutations, which occur coincidally or
      subsequently to driver mutations. These mutations are defined to not
      contribute to the selective fitness of the cell, even though this
      conception has been challenged by stochastic tumor progression simulations
      {\cite{stochastic_cancer}}. Some studies have reported that cancer cells
      carry 40--80 somatic mutations, and only 5--15 of them are driver
      mutations {\cite{som_mut}}.

      Estimating the number of somatic driver and passenger mutations and the
      rate at which they occur is not well established {\cite{driver_passenger}.
      Two tumors, even though histologically indistinguishable, might present
      different subsets of mutations {\cite{driver_passenger}
      {\cite{intertumor}}. This observation has been defined as inter-tumor
      heterogeneity. Additionally, tumors present heterogeneity at the
      intra--tumor level {\cite{intratumor}: subclones of the tumor might
      present different mutations.

      \paragraph{Tumor suppressor} genes protect a cell from entering the path
      to cancer. They comprise genes encoding for cell adhesion proteins,  DNA
      repair proteins, proteins acting in apoptosis pathways, or cell cycle
      proteins {\cite{tumor_supp}}. The action of these proteins inhibits metastasis, excessive cell
      survival or proliferation. Tumor suppressors mostly follow the two-hit
      hypothesis {\cite{two_hit}}: to inactivate the tumor-protecting role of tumor
      suppressors, two genetic events, often LOH in  combination with silencing
      point mutations or silencing of both alleles by somatic events, are
      necessary to inactivate both alleles of the gene.
      Compared to dominant oncogenes, tumor suppressor genes are often
      considered to be recessive. Alternatively, tumor progression can be
      influenced by functional haploinsufficiency of tumor suppressors {\cite{haploinsuff}}.
      According to this conception, a disease state can emerge if a cell /
      organism has only one functional copy of a given gene and if it cannot
      produce enough of a gene product to establish a wild-type condition. APC
      and TP53 are amongst the best known tumor suppressors.

      In the canonical Wnt signaling pathway, a destruction complex, including
      APC, leads to $\beta$--catenin phosphorylation, followed by
      ubiquitination, marking it for degradation in the proteasome. Activation
      of Wnt signaling inhibits the destruction complex.  Consequently,
      $\beta$--catenin is no longer marked for degradation and can translocate
      to the nucleus, where it acts on gene expression of target genes {\cite{wnt_signal}}. In many
      tumors, loss or dysfunction of APC leads to $\beta$--catenin accumulation
      in the nucleus even in the absence of an extracellular stimulus,
      resulting in increased cell migration and decreased cell adhesion and
      apoptosis {\cite{wnt_signal_2}}.

      TP53 is the master guardian of the genome {\cite{tp53_1}}. In normal situations, p53, the
      protein encoded by TP53, is targeted for ubiquitination and degradation in
      the proteasome {\cite{tp53_2}}. In case of cellular stress, p53 is no longer
      ubiquitinated. p53 can then stop the cell cycle at the G1/S and G2/M
      transitions, induce DNA repair, and induce apoptosis if the damage cannot
      be repaired {\cite{tp53_3}}. One mechanism by which p53 acts on cell-cycle arrest is by
      activating expression of p21. p21 binds to the G1/S transition complex and
      inhibits its activity, leading to cell-cycle arrest {\cite{tp53_3}}. Inactivation or
      mutation of TP53 is a crucial step in many cancers, leading to a loss of
      control over DNA stability {\cite{tp53_4}}.

      \paragraph{Oncogenes} comprise several GTPases, transcription factors,
      receptor tyrosine kinases and growth factors {\cite{oncogenes}}. Overexpressed or overactive
      versions of these proteins lead to increased mitogenic signals, causing
      increased cell growth or proliferation. Two important oncogenic pathways
      include the RAS--RAF--MEK--ERK and PTEN--PI3K--AKT pathways, which can
      both be activated by ligand--binding on Epithelial Growth Factor. EGFR is
      a cell surface tyrosine kinase receptor {\cite{egfr_review}}. It is anchored in the cytoplasmic
      membrane and is composed of an intracytoplasmic tyrosine kinase domain, a
      short hydrophobic transmembrane domain and an extracellular ligand-binding
      domain {\cite{egfr_review_2}}. Ligand binding causes a conformational change of the receptor,
      which leads to homo-- or heterodimerization, followed by an auto-- and
      cross--phosphorylation of key tyrosine residues on its cytoplasmic domain {\cite{egfr_review_2}}.
      This forms docking sites for cytoplasmic adaptor proteins that contain
      phosphotyrosine-binding and Src homology 2 domains.

      Signaling through the PI3K--AKT pathway leads to cell growth,
      proliferation and survival. The signaling cascade is initiated by
      integrins, cytokine receptors, G--protein coupled
      receptors, and receptor tyrosine kinases, such as EGFR {\cite{pi3k_2}}. Activation of the
      receptor results in production of PIP3 by activation of PI3K. PIP3 is
      anchored in the cell membrane and acts as docking site for proteins
      containing PH domains, such as PDK1. PIP3-bound PDK1 partially activates
      Akt by phosphorylation {\cite{pi3k_3}}. Full activation of Akt is achieved by
      phosphorylation of PDK1 by mTORC2 {\cite{pdk1}}. Activated Akt then acts on a variety of
      proteins necessary for protein synthesis, glucose metabolism, cell
      survival / death and proliferation. The phosphatases PP2A and PHLPP can
      dephosphorylate and thereby inactivate Akt {\cite{pdk1}}. Additionally, PTEN
      dephosphorylates PIP3 and indirectly also inactivates Akt {\cite{pdk1}}. Dysregulation
      of the PI3K--AKT has been associated with several human diseases including
      neurological diseases, diabetes and cancer {\cite{akt}}. In cancer, inactivation of
      PTEN and kinase activity activating mutations on PI3K and Akt are found
      recursively, leading to enhanced signaling, leading to inhibition of
      apoptosis and increased proliferation {\cite{pi3k}}.

      In the RAS--RAF--MEK--ERK pathway, ligand binding on cell surface receptor
      tyrosine kinases activates the receptor. One of these receptors is the
      EGFR. GRB2 binds to Tyr1068 of EGFR through its SH2 domain and recruits
      SOS, a guanine nucleotide exchange factor {\cite{grb2}}. Grb2 and SOS then form a
      complex with the activated EGFR, which activates SOS {\cite{grb2}}. Activated SOS
      promotes recruition of Ras proteins to the activated EGFR. Through its GEF
      activity, SOS then induces GDP removal from Ras proteins, which can
      subsequently bind GTP and become active. Activated Ras recuits Raf
      proteins to the cell membrane and binds to their N-terminus. The
      activation of Raf, serine/threonine kinase proteins, is complex. In fact,
      Raf proteins are considered as gatekeepers of the RAS--RAF--MAPK pathway.
      In its inactive form, Raf is present in a 'closed' conformation, in which
      an autoinhibitory domain blocks the catalytic kinase domain {\cite{raf}}. Recruitment
      to the cell membrane of Raf by Ras results in a conformational change {\cite{raf_2}},
      which disrupts the autoinhibitory interaction of Raf. Rafs then form
      homo-- or heterodimers, which leads to partial activation by allostery.
      Transphorylative events, with optional phosphorylation by other kinases,
      such as PAK1 {\cite{pak1}}, then fully activates Raf. Activated Raf can now bind to
      MEKS, which are tyrosine/threonine kinases. MEKs phosphorylate ERKSs,
      which are also serine/threonine kinase enzymes. ERKs then translocate to
      the cell nucleus, where they influence expression of target genes {\cite{pak1}}.
      RAS--RAF--MEK--ERK signaling promotes cell-cycle progression, cell
      differentiation, growth and survival {\cite{pak1}}.

      \begin{figure}[ht]
        \begin{center}
          \includegraphics[scale=2.5,angle=0]{egfr_signaling.png}
          \caption{Schematic representation of the EGFR signaling cascade}}
        \end{center}
      \end{figure}

      \subsubsection{Molecular Profiling of Solid Tumors}

      Classical anti-cancer treatments are tailored for the „average patient“
      and not for the individual. Traditional cytotoxic chemotherapeutic drugs,
      for instance, are unspecific and have numerous adverse effects: they
      attack rapidly dividing cells and make no difference between healthy and
      cancer cells. For a long time, there was no possibility to predict the
      success of a patient’s cancer treatment. In consequence, the clinician had
      no way to personalize the treatment to the individual patient. Molecular
      profiling of tumors by several methods has lead to a better understanding
      of cancer development and progression and to the identification of some
      recursively found driver mutations, which may be potential anti-cancer
      targets.

      Lung cancer, melanoma and colorectal cancer are amongst the most common
      cancers worldwide. Classically, diagnosis has been made by observing
      histologic, anatomic and pathologic alterations. Molecular
      profiling of tumors by several methods has lead to a better understanding
      of cancer development and progression and to the identification of some
      recursively found driver mutations, which may be potential anti-cancer
      targets.

      \textbf{Lung cancer} is the most common cancer worldwide, both in terms of
      new cases (1.8 million) and deaths (1.6 million) (cancer.org). Smoking is a widely
      accepted risk factor, as chemical carcinogens in tobacco smoke induce
      several genetic mutations {\cite{smoking_cancer}}. Lung cancer can be divided into two
      histological subtypes: small-cell lung cancer (SCLC) and non-small cell
      lung cancer (NSCLC). Over the last decade, it has become clear that these
      subtypes can be classified into  additional classes by the mutational
      status of recurrent driver mutations.

      A combination of oncogenic triggers cause cells of the normal bronchial
      epithelium to proliferate, giving rise to meta--, hyper-- and dysplastic
      epithelial lesions. Genomic events in early stages of lung cancer giving
      rise to atypical adenomatous hperplasia include LOH on chr.3p,
      p16\textsuperscript{INK4a} or RB inactivation, as well as mutations in
      KRAS or in $\beta$--catenin. TP53 inactivation and LOH on chr.13q are
      believed to favor progression into the primary adenocarcinoma stage. After
      that stage, major chromosomic instability is often detected, giving rise
      to metastatic adenocarcinoma. These chromosomic events include LOH on
      chr.2q, chr.9p, chr.18q, and chr.22q. Additionally, the oncogene c-myc can
      be amplified in late stages. {\cite{nsclc}}

      Frequent mutations in NSCLC affect EGFR (10--35\%), KRAS (15--25\%), PTEN
      (4--8\%), HER2 (2--4\%), DDR2 (4\%), PIK3CA (1--3\%), BRAF (1--3\%), AKT1
      (1\%), MEK (1\%) and NRAS (1\%). Additionally, rearrangement of ALK
      (3--7\%), RET (1\%) and ROS1 (1\%) and amplifications of FGFR1 (20\%) and
      MET (2--4\%) are found recursively. These mutations are rarely observed
      together in the same tumor. (mycancergenome.com)

      \textbf{Melanoma} develops from the malignant transformation of melanocytes
      in the basal epidermal layer of the skin. Exposure to UV light,
      immunosuppression, fair-skin and multiple nevi are risk factors. UV
      radiation causes cyclobutane pyrimidine dimers (CPDs) {\cite{melanoma_3}}. T--T, C--C or C--T
      dimers (UV fingerprints) are formed, leading to direct DNA damage. People
      diagnosed with rare genetic disorders like xeroderma pigmentosum are at
      great risk {\cite{xero}}. Traditionally, melanoma has been classified based on
      histological and pathological properties, such as the thickness of the
      tumor, ulceration or the anatomic location of the tumor.

      Spontaneous mutations in BRAF and NRAS and epigenetic modulations of APC
      are believed to promote progression of normal epithelium into a dysplastic
      nevus. Mutations affecting genes PTEN and CDKN2A and altered gene
      expression of MGMT or RASSF1 then favor invasion of the dermis, which is
      underlying to the epidermis, during the radial growth phase. Genetic and
      epigenetic alterations in AKT, MTAP, APAF, CDH1 and CDH2 then result in
      the vertical growth phase, where the tumor invades surrounding tissues.
      Finally, loss of TP53 functionality and further epigenetic modulations of
      MTA2 and MAGE enable metastasis. {\cite{melanoma}} {\cite{melanoma_2}}

      The occurrence of the different mutations differs by the anatomic location
      of the tumor, e.g. whether the specific body part is chronically exposed
      to the sun. Mutations frequently found in melanoma occur on BRAF
      (37--50\%), NRAS (13--25\%), MEK (6--7\%), NF1 (11.9\%), CTNNB1 (2-4\%),
      GNAQ (1.3\%) and GNA11 (1.2\%). (mycancergenome.com)

      \textbf{Colorectal cancer (CRC)} is one of the best studied cancers. The
      development of colorectal adenocarcinomas occurs over many years. Caused
      by the acquisition and accumulation of driver mutations, a normal
      colorectal epithelium can progress to adenoma, which develops into
      carcinoma, which can eventually mestastasize.

      The molecular progression models in CRC depend on the underlying
      instability process (chromosomal vs. microsatellite). In CIN CRCs, loss of
      the tumor suppressor gene APC is often causing the evolution from a normal
      to a hyperproliferative epithelium. Progression to adenoma stages are
      associated with DNA hypomethylation, KRAS activation and loss of 18q.
      Mutations in TGF$\beta$RII and PIK3CA and loss of TP53 by LOH on chr.17p
      then lead to the final carcinoma stage. In MSI CRCs, mutations and
      hypermethylations in MMR genes result in an hyperproliferative epithelium.
      BRAF mutations, followed by PIK3CA mutations, loss of TP53 and frameshift
      mutations affecting TGF$\beta$RII, BAX or IGF2R are associated with CRC
      progression towards the carcinoma stage. {\cite{cin_crc}} {\cite{crc}} {\cite{crc_2}}

      Mutations recursively detected in CRC occur on KRAS (36--40\%), SMAD4
      (10--35\%), PIK3CA (10--30\%), BRAF (8--15\%), PTEN (5--14\%), NRAS
      (1--6\%), and  AKT1 (1--6\%). (mycancergenome.com)

      The mentioned molecular progression profiles are likely to be an
      oversimplification. These models are based on frequently found alterations
      in the respective cancers. Due to tumor heterogeneity, these alterations
      do not always have to be observed within the tumor, and the chronological
      appearance of these alterations may vary from one tumor to another.
      Pharmacoligically, only a few of these driver mutations are clinically
      actionable, e.g. can be targeted with drugs.

      Despite enormous advantages in the understanding of the underlying causes
      of cancer in the last decades, only a few proteins involved in cancer
      development and progression are clinically actionable, e.g. can be
      targeted with specific drugs.

  \subsection{Targeting the EGFR signaling pathway}

    \subsubsection{Biological Role of EGFR in Solid Tumors}

      The EGFR pathway is long known to be dysregulated in most solid tumors
      and thereby presents a rational target for cancer therapy.
      In normal cells, the tighly regulated EGFR signaling pathway drives
      cell-cycle progression, affects differentiation and migration and acts as
      a survival signal. EGFR ligand binding leads to an activation of several
      signaling cascades, such as the previously discussed PI3K--AKT and
      RAS--RAF--MEK--ERK pathways. EGFR levels have been shown to be
      higher in tumor samples than in surrounding tissues. Also, more of EGFR's
      ligands EGF and TGF$\alpha$ are found in these locations. EGFR
      overexpression may result from epigenetic alterations or gene
      amplification. Increased EGFR mRNA levels in in vitro culture of human CRC
      cells have also linked EGFR overexpression to tumor progression. EGFR
      overexpression has been associated with poor prognosis if treated with
      classical chemotherapy, as this leads to a more aggressive progression.

      \paragraph{EGFR signaling as a survival signal: }

        Cells communicate with their environment. Without extracellular signals,
        a cell undergoes apoptosis. Signaling pathways induced by cell--matrix
        interactions, cell-cell interactions, and soluble survival factors act
        on a variety of genes and proteins. Loss of matrix attachment leads to
        cell growth arrest and even cell death in normal epithelial cells, a
        process called anoikis. The cell--matrix interaction provides important
        spatial informations to the cell and acts as a safeguard against
        inappropriate proliferation and migration.

        Activation of the EGFR signaling pathway protects normal
        epithelial cells against anoikis in the suspended state. EGFR blockade
        sensitizes normal epithelial cells to apoptosis, but the effect is much
        more pronounced in the suspended state than in the attached state. The
        redundancy of cell survival signals makes normal epithelial cells
        relatively resistant to EGFR-blockade in their normal microenvironment.

        Tumor cells are often in transit or at sites with inadequate matrix
        composition. They are often provided with inadequate or missing
        cell--cell and cell--matrix interactions. They are thereby more dependent
        on survival signals propagated by soluble mediators, such as EGF or
        TGF$\alpha$. Consequently, tumor cells are relatively sensitive against
        blockade of EGFR or other cell surface receptors. This is counterbalanced by an upregulation of cell
        surface receptors that activate anti--apoptotic pathways. MAPK
        activation in the EGFR--activated RAS--RAF--MAPK--ERK pathway is
        required for high expression of Bcl--XL, an anti-apoptotic protein of
        the Bcl--2 family. Bcl--2 proteins can be either pro-- or
        anti--apoptotic. They regulate liberation of cytochrome c from the
        mitochondria, which is essential in the apoptotic caspase pathway.
        Additionally, EGFR signaling leads to post-transcriptional
        phosphorylation on the pro-apoptotic Bad protein, which is thereby
        functionally inactivated.

    \subsubsection{EGFR-targeted drugs}

      The observations that EGFR is recursively upregulated in many cancers
      and that EGFR is such an important mediator of cell-cycle,
      progression, cell growth, and cell survival, has lead to the
      development of agents that block this pathway. Pharmacologically,
      these agents can be classed by their mode of action: EGFR-targeted
      monoclonal antibodies and EGFR-specific tyrosine kinase inhibitors (TKIs).
      Several proteins acting downstream of EGFR are often
      found to be mutated. This lead to the development of other targeted
      drugs, such as BRAF-- or MEK--inhibitors. Table 1 shows a selection
      of FDA-approved cancer drugs that target components of the EGFR signaling
      pathway.

      \begin{table}[!htbp]
          \caption[Targeted Cancer Agents]{FDA-approved cancer drugs for solid tumor treatment that target the EGFR pathway}
          \centering
          \begin{tabular}{ |p{4cm}|p{3.7cm}|p{6.3cm}|}
          \hline
          Agent & Target(s) & FDA-approved indication(s) \\ \hline \hline
          Afatinib (Gilotrif) & EGFR & NSCLC (with EGFR del19 or L858R) \\
          Cetuximab (Erbitux) & EGFR & Colorectal cancer (KRAS WT) \\
          Cobimetinib (Cotellic) & MEK & Melanoma (with BRAF V600E or V600K \\
          Dabrafenib (Tafinlar) & BRAF & Melanoma (with BRAF V600 mutation) \\
          Erlotinib (Tarceva) & EGFR & NSCLC \\
          Gefitinib (Iressa) & EGFR & NSCLC (with EGFR del19 or L858R) \\
          Necitumumab (Portrazza) & EGFR & Squamous NSCLC \\
          Osimertinib (Tagrisso) & EGFR & NSCLC (with EGFR T790M) \\
          Panitumumab (Vectibix) & EGFR & Colorectal cancer (KRAS WT) \\
          Trametinib (Mekinist) & MEK & Melanoma (with BRAF V600) \\
          Vemurafenib (Zelboraf) & BRAF & Melanoma (with BRAF V600) \\
          \hline
        \end{tabular}
      \end{table}

      Anti--EGFR monoclonal antibodies bind to the extracellular domain of EGFR
      in its inactive state and thereby compete for EGF or TGF$\alpha$ binding.
      They thereby inhibit ligand-induced EGFR tyrosine kinase activation. The
      most popular anti--EGFR monoclonal antibody is cetuximab. Cetuximab binds
      to EGFR with a higher affinity than the natural ligands EGF or
      TGF$\alpha$. Cetuximab binding induces internalization of EGFR, following
      by its degradation. Cetuximab also binds to EGFRvIII, a constantly active
      version of EGFR. Cetuximab blocks cell--cycle  progression at the G0/G1
      boundary, inhibits cell proliferation and induces cancer cell death.

      EGFR-specific tyrosine kinase inhibitors comprise three classes that
      include first generation reversible  EGFR-inhibitors (gefitinib,
      erlotinib), second generation irreversible inhibitors (afatinib,
      dacomitinib, neratinib) and third generation mutant-selective inhibitors
      (brigatinib, osimertinib, rociletinib). Third generation agents have a
      better sensitivity against mutated than wild-type EGFR and have been
      designed to further decrease treatment-associated side effects. TKIs
      are low molecular weight molecules that are mainly
      derived from quinazoline. These compounds bind to EGFR and block
      ligand-induced receptor phosphorylation by occluding the ATP--binding
      site. This results in inhibition of cell proliferation, cell--cycle arrest
      at the G0/G1 boundary and apoptosis.

    \subsubsection{Predictive markers}

      Mutations in EGFR and downstream proteins are predictive of the potential
      success of EGFR-targeted therapy. In many cases the predictive value of a
      marker, even though theoretically reasonable, has not yet been
      established. Demonstration of the predictive value of these markers is not
      trivial and has to be proved in clinical trials. The search for
      additional predictive markers is ongoing and many EGFR-targeted
      agents are still in clinical trials.

      \paragraph{EGFR} is a strong predictive biomarker for the success of the
      administration of EGFR-specific tyrosine kinase inhibitors. EGFR
      activating mutations are observed in 10--35\% of NSCLC. 90\% of EGFR
      mutations are exon 19 deletions (48\%) and exon 21 L585R (c.2573T$>$G)
      (43\%) point mutations. In melanoma and CRC, EGFR mutations are seldom.
      These mutations confer increased sensitivity to EGFR--specific tyrosine
      kinase inhibitors. Patients with EGFR--mutated tumors have a longer
      progression--free survival than those treated with traditional
      chemotherapy. Also, patients with EGFR--mutated tumors display a better
      prognosis if treated with EGFR TKIs compared to patients with wild--type EGFR
      cancers.

      \paragraph{KRAS} mutations are found in 36--40\% of CRCs, 15--25\% of
      NSCLC, and in 2\% of melanomas. Critical mutations in the KRAS gene
      include mutations in codons 12, 13 and 61. Amongst these, the G12C variant
      is the most common. These mutations lock KRAS in its GTP-bound state,
      resulting in a constantly active protein. This then leads to a constantly
      active signal transduction. Blocking EGFR in that case is useless, as KRAS
      acts downstream of EGFR. Several KRAS point mutations in codons 12, 13 and
      61 have been shown to confer reduced sensitivity to EGFR-targeted
      monoclonal antibodies in CRC and EGFR--TKIs in NSCLC.

      \paragraph{NRAS} is an isoform of KRAS. Activating mutations in NRAS
      codons 12 and 61 are found in 1--6\% of CRCs, 13--25\% of melanomas and
      1\% of NSCLCs. NRAS mutations have been associated with reduced
      sensitivity to EGFR monoclonal antibodies in CRC. The predictive value of
      the influence of NRAS mutations in NSCLC and
      melanoma is unknown at this time.

      \paragraph{BRAF} mutations are very common in melanoma (37--50\%) are
      found in 8--15\% of CRCs and 1--4\% in NSCLCs. Amongst BRAF-mutated
      melanomas, the V600E variant is found in 80--90\% cases. The V600E variant
      occurs in the activation segment of the BRAF kinase domain and results in
      increased kinase activity. BRAF mutations usually confer a resistance to
      EGFR-targeted therapy in KRAS WT tumors. BRAF V600E mutations have been
      associated with increased sensitivity to BRAF inhibitors in melanoma and
      NSCLS and MEK inhibitors in melanoma.

  \subsection{Tumor DNA Sequencing}

    The completion of the Human Genome Project in 2001 resulted in a massive
    boost in molecular medicine. New high-throughput techniques, in combination
    with advanced computational performance and storage capacities, lead to an
    explosion of biological data. Amongst the many mutation detection
    techniques, Next--Generation Sequencing (NGS) constitutes the most powerful
    method and allows deep insights into the underlying causes of diseases.
    Today, NGS is used in several disciplines, which include basic molecular
    biology and pharmacogenomic research, forensics and molecular diagnostics.
    Even though advances in sequencing technology and computational power and
    tools have decreased the time and cost of a sequencing experiment, NGS is
    still mainly used in research, with only a few laboratories using this
    technique in diagnostics.

    NGS has profoundly impacted the field of oncology. A wide variety of NGS
    applications have been applied to study the genetics and epigenetics of
    cancer. ChIP--Seq and FAIRE--Seq allow determination of DNA--protein
    interactions and identification of DNA regulatory elements, respectively.
    The cancer transcriptome can be studied with RNA--Seq experiments. NGS has
    accelerated discovery of genetic and epigenetic alterations in tumors. Also,
    the time and cost of an NGS experiment are rapidly decreasing.

    Targeted NGS is the method of choice in molecular pathology laboratories.
    This method allows deeper insights into oncogenes and tumor suppressor genes
    than array--based and single--gene approaches, which are currently used in
    most laboratories. Multiple genes can be studied in a single experiment,
    while classical methods are much more restricted in that regard. Targeted
    NGS experiments differs from whole--genome or whole--exome sequencing, as
    they capture and sequence only a selection of regions of interest (ROIs).
    This approach increases the efficiency of the experiment and allows to
    sequence more samples in the same period of time. Additionally, coverage,
    e.g. the number of sequencing reads that align to a specific base of the
    reference genome, is drastically increased in targeted NGS. This offers the
    sensitivity to detect low--frequency mutations in the tumor sample.
    Targeted NGS methods can be applied to study insertions, deletions and
    point mutations in genes of interest. NGS can thereby guide personalized
    cancer therapy by identifying the mutational status of genetic markers,
    which are predictive of the potential success of targeted cancer treatment.

    \subsubsection{Practical implications in the laboratory}

      The implementation of new techniques into the workflow of molecular
      diagnostics laboratories requires a careful assessment of the sensitivity
      and sensibility of the method. The quality of the genetic testing of the tumor is affected by several
      factors. These include the content of tumor cells in the sample, the
      quality of the tissue material, sequencing library preparation and the the
      bioinformatic pipeline.

      The biopsy usually consists of an admixture of normal and cancer cells.
      The sensitivity of tumor variant detection is linked to the tumor cell
      content of the specimen. In addition, cancers are highly heterogenous,
      e.g. a small subpopulation might present mutations that provide resistance
      to the treatment. Detecting these low-frequency mutations and clearly
      delineating them from possible sample processing or sequencing induced
      artifacts presents an important challenge.

      Tumor biopsies usually yield a limited amount of tissue, therefore it is
      important to optimize sample usage by multiplexing analysis. In
      Luxembourg, all relevant tumor biopsies are usually sent to the
      Laboratoire National de Santé (LNS) to the Service of Pathologic Anatomy
      where the biopsy is fixed in formalin and embedded in paraffin (FFPE).
      FFPE preserves the tissue morphology and thereby allows histological
      analysis. In addition, it allows specimen storage for decades. Sample
      quality, however, is influenced by this fixation method and the fixation
      time. DNA extraction from FFPE samples is difficult and yields low amounts
      of DNA; formaldehyde leads to cross-linking of nucleic acids and proteins;
      FFPE introduces fixation artifacts into DNA sequences, for instance C>T
      transitions. These circumstances complicate sample processing as well as
      NGS data interpretation. Though, FFPE samples have been shown to be still
      suitable for downstream analyses.

      Several NGS bench-top devices have become available in the last decade.
      These instrumentations differ in their underlying chemistry that
      influences the instrument’s performance, accuracy, output and time per
      run. Common sequencing principles include pyrosequencing (454), sequencing
      by ligation (SOLiD), ion semiconductor sequencing (Ion Torrent) and
      sequencing by synthesis (Illumina).

      Sequencing library preparation also affects the final NGS result. Several
      technologies for target enrichment exist and are available for different
      sequencing instruments. Two steps are essential for all these enrichment
      methods: target regions have to be enriched and samples have to be
      multiplexed, which requires the incorporation of a unique index adaptor
      combination for each sample. Target enrichment methods can be separated
      into three basic groups: targeted circularization, hybrid capture of
      target fragments and PCR-based enrichment methods. Short-range multiplex
      PCR produces short DNA fragments of target regions in a first PCR. In a
      second PCR, adapters and indices are added. Hybridization-based methods
      require a so-called shotgun library construction before target regions can
      be captured. During this process, genomic DNA is sheared randomly into
      small fragments and an adapter- and index-linked library is produced.
      Biotinylated baits are added that bind to target regions. Target regions
      can then be captured using streptadivin coated magnetic beads. Targeted
      circularization methods rely on a digestion of DNA by restriction enzymes.
      The produced DNA fragments are then circularized and only circularized
      target regions are amplified by PCR.

      The establishment and validation of a bioinformatic NGS data analysis
      pipeline still constitutes a challenge in diagnostics. After generation of
      FASTQ files of the sequencer, data generally undergo quality control,
      followed by trimming of low quality bases, alignment to the reference
      genome, variant calling and variant annotation. For each of these steps,
      several bioinformatic algorithms and tools exist. The computational
      pipeline of the molecular pathology laboratory has to incorporate the
      tools that allow the most sensitive and sensible analysis of data. For
      instance, quality trimming influences the mapping to the reference genome.
      The mapping, in turn, strongly affects the variant calling. In fact,
      variant calling is a critical step in NGS data analysis. Several tool kits
      as SAMtools, SPLINTER, VarScan2 or GATK allow variant annotation, but vary
      in their false-positive and false-negative detection rates.
      These tools have to be carefully assessed, as false-positives or
      false-negatives should absolutely be avoided when it comes to the
      subscription of a targeted chemotherapeutic agent.

      To facilitate interpretation of NGS data, variants have to be annotated
      and their clinical actionability has to be identified. Several databases
      have emerged in this field  (such as mycancergenome.org) and numerous
      tools allow to automatize variant annotation. Here again, the choice of
      the database and the variant annotator is important.

      Finally, the sample-to-results time is a very pragmatic, but important
      factor. The time from the biopsy to the potential start of an
      administration of a targeted chemotherapeutic drug should be reduced to a
      minimum. For instance, in case of late-stage cancer patients, it would be
      unacceptable if analysis would take several weeks. To reduce the
      sample-to-results time to under two weeks, the sample processing workflow
      should be as short as possible, while still yielding high quality
      sequencing libraries. The bioinformatic pipeline should not only
      incorporate the best tools, but should also be automatized to further
      reduce the time of analysis.

  \subsection{Aims of the Thesis}
