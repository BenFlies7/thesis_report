\section{Introduction}

  Cancer represents a huge burden for health care systems worldwide and is one
  of the leading death causes. Scientific discoveries in the last decade have
  had an enormous impact on our understanding of the underlying causes of
  cancer. The development of omics techniques, in combination with advanced
  computational power, has lead to an explosion of biological data. It has
  become clear that cancer is an incredibly complex malignancy, which is
  affected by genetic, environmental and behavioural factors. The research
  community is trying to interprete this vast amount of data with the goal to
  get a deeper understanding of cancer and to cure it eventually. In recent
  years, several drugs have been approved, which target proteins needed for
  cancer development, proliferation or metastasis. Molecular testing is employed
  to check whether these targeted drugs would be of benefit. In that regard,
  Next-Generation Sequencing (NGS) is an interesting method to gain deep
  insights into the genetic information of a tumor and to guide personalized
  therapy.

  \subsection{Cancer Genetics}

    DNA in virtually all cells undergoes continuous damage. In normal cells, this
    damage is repaired without without errors. In cancer cells, the equilibrium
    between DNA damage and repair systems is dysbalanced, leading to a mutator
    phenotype. The resulting genomic instability manifests itself in an
    accumulation of mutations.

    The genetic diversity caused by this instability, the cardinal feature of
    cancer, in combination with several environmental factors, such as
    inflammation, enables the hallmarks of cancer (Hallmarks of Cancer: The Next
    Generation). These include replicative immortality, cell death resistance,
    ongoing proliferative signaling, invasion and metastasis, growth suppressor
    evasion, inducement of angiogenesis, energy metabolism reprogramming and
    immune destruction evasion.

    \textbf{Mutations caused by DNA damage:} DNA can be damaged by endogenous
    and environmental agents. Carcinogenic substances produced by industry or
    present in tobacco smoke increase cancer risk. Cellular metabolic processes
    also produce DNA-damaging products, such as reactive oxygen species. Several
    cellular DNA repair systems have emerged, such as the nucleotide excision or
    base excision pathways. DNA lesions can escape these repair mechanisms if
    the damage happens in an inaccessible region of the DNA. Also, the repair
    systems cannot cope with the rate of mutation if the frequency at which they
    occur becomes to important. These DNA lesions, if not repaired, then induce
    errors in the replication by DNA polymerases.

    \textbf{Mutations caused by misincorporation by DNA polymerase:} The process
    of DNA replication is not free from errors. It has been estimated that DNA
    polymerase has error rates ranging from 10\textsuperscript{-4} to
    10\textsuperscript{-6}. This is followed by mismatch repair, which is often
    intrinsic to the polymerase. Their 3'->5' exonuclease activity allows
    excision of 90-99[^\% of misincorporated bases, decreasing the overall error
    rate to 10\textsuperscript{-6} -- 10\textsuperscript{-8}. Additionally,
    several DNA polymerases exist, which differ in their error rates. DNA
    polymerase $\beta$ has a much worse error rate than DNA polymerase $\delta$
    or $\epsilon$. There is evidence that these enzymes can be used
    interchangeabily, and that DNA $\beta$ is increased in some tumors. This
    leads to more errors, resulting in increased mutagenesis.

    \subsubsection{Genomic instability in cancer}

      \textbf{Chromosomal instability (CIN)} is the most common kind of
      instability in solid tumors and is one of their hallmarks. It has been
      postulated that chromosome missegregation plays a crucial role in cancer
      adaptation. Chromosome segregation during mitosis requires proper
      attachment of microtubules at the kinetochores. Several proteins are
      involved in this process. Defects in these proteins lead to chromosome
      missegregation. This leads to telomere dysfunction,  faulty sister
      chromatid cohesion, loss of heterozygosity (LOH), hypo-- or hyperactive
      spindle assembly checkpoint or defective centrosome duplication. Another
      chromosomal instability process has been described recently:
      chromothripsis happens when chromosomes are fragmented. The cell tries to
      repair the chromosomes, but this process is far from being perfect,
      leading to massive chromosomal rearrangements. About 70\% of solid tumors
      are aneuploid. Aneuploidy has been correlated to increased tumor
      aggressiveness and advanced cancer stages. The question whether CIN is a
      cause or consequence of tumor development remains unanswered.

      A common observation caused by CIN is the loss of 18q, which includes
      genes coding for SMAD2, SMAD4 and DCC. 18q LOH has been associated
      with poor survival. DCC is a cell surface receptor and
      is involved with apoptosis and cell-adhesion. Both SMAD2 and SMAD4 are acting
      in the TGF-b signaling pathway, which has tumor-suppressing properties.
      LOH of these genes, followed by a somatic inactivating point mutation,
      leads to the inactivation of these genes, thereby vanishing the
      tumor protecting role of these proteins ('two-hit hypthesis'). Similarly,
      the loss of 17p affects the important tumor suppressor gene TP53,
      the 'guardian of the genome'.

      \textbf{Microsatellite instability (MSI)} is less common than CIN, but
      still occurs in 15\% of CRCs.  Microsatellites are short DNA segments with
      tandem repeats. About 500,000 of microsatellites are estimated in the
      human genome. Microsatellite elongation or shortening is a consequence of
      defective or inactive DNA mismatch repair (MMR), which corrects base
      replication errors. Seven enzymes contribute to the MMR system (MLH1,
      MLH3, MSH2, MSH3, MSH6, PMS1, PMS2). Germline mutations in MMR genes cause
      the Lynch syndrome (hereditary nonpolypsos colorectal cancer). Patients
      have an 80\% lifetime risk to develop colon cancer.Germline LOH of one
      allele with somatic inactivation on the other allele or double allelic
      inactivation by somatic mutations of these genes can cause MSI. The most
      common reason for MMR inactivation is through methylation of the promoter
      of the MLH1 gene. DNA polymerase has a higher error rate in repetitive
      regions. When MMR genes are inactivated or defective, the replication
      mistakes in microsatellites cannot be corrected: MSI is the consequence.
      In some cancers, MSI can occur despite functional MMR through frameshift
      mutations at microsatellites. MSI is often associated with cancers
      harboring mutations in TGFβRII, EGFR, PTEN, and BAX, which contain such
      simple repeats.

      prognosis

      \textbf{CpG island methylator phenotype (CIMP)} is caused by epigenetic
      processes. CpG islands are regions with high CpG dinucleotide content.
      CpG islands are common in gene promoters. In normal tissue, CpG islands in
      gene promoters tend to be not methylated. In cancer cells, CpG islands are
      often found to be methylated. DNA methylation is an epigenetic process,
      that, if it occurs in gene promoters, leads to transcriptional silencing
      of genes. Several apoptosis, DNA repair, tumor suppressor and cell cycle
      genes have been found to be silenced by DNA methylation in cancer cells.

      prognosis


    \subsubsection{Mutator phenotype}

    \subsubsection{Tumor heterogeneity}

    Tumor instability and the accumulation of mutations presents a major problem
    in cancer therapy. Chemotherapy, for instance, can induce an environment where
    chemotherapy sensitive cells are killed, while resistant cells can survive.
    These resistant clones are often the source of relapses and the
    chemotherapeutic agent will lose its efficiency. It is thereby of utmost
    necessity to hit the tumor as fast and as hard as possible. Due to the
    cancer's heterogeneity, a combination of several therapeutic methods and
    agents often yields the best results.

    \subsubsection{Driver and passenger mutations}

      Genomic instability in cancerous cells becomes a critical mechanism if it
      affects oncogenes or tumor suppressor genes, which have the potential to
      be tumor 'driver' mutations. Many alterations found in cancer cells are
      passenger mutations, e.g. do not contribute to the selective fitness of
      the cell. Driver mutations, often happening on oncogenes or tumor
      suppressor genes, promote the cell's fitness. The heterogeneous population
      of cancer cells harbors cells with different random somatic and
      non-deleterious mutations. Each cell exhibits a unique combination of
      genetic and environmental perturbations. Cells are in competition for a
      variety of resources, which include space, oxygen and nutrients.
      Eventually, cells with the best fitness, e.g. with the highest
      proliferative potential  and the lowest death rate, are then selected
      through natural selection principles. These cells will outlast less fit
      cells. This results in sequential waves of clonal expansion, leading to
      different subclones within the same tumor that differ in their
      proliferative, migrative and invasive potential. The hypothesis that
      passenger mutations, that occur subsequently or coincidentally to driver
      mutations, do not influence the cell's fitness at all has been challenged
      by stochastic tumor evolution simulations. Since then, it has been
      proposed that, even though the individual effect may be small, the
      cooperation of multiple accumulated small-scale passenger mutations might
      play present role in cancer development and progression.

      \paragraph{Tumor suppressors}

      Tumor suppressor genes protect a cell from entering the path to cancer.
      They comprise genes encoding for cell adhesion proteins,  DNA repair
      proteins, proteins acting in apoptosis pathways, or cell cycle proteins.
      The action of these proteins inhibits metastasis, excessive cell survival
      or proliferation. Tumor suppressors mostly follow the two-hit hypothesis,
      which was first proposed by Knudson  for the retinoblastoma protein (pRb):
      to inactivate the tumor-protecting role of tumor suppressors, two genetic
      events, often LOH in  combination with silencing point mutations or
      silencing of both alleles by somatic events, are necessary to inactivate
      both alleles of the gene. Another possibility of tumor suppressor
      inactivation is methylation of the gene promoter. Compared to dominant
      oncogenes, tumor suppressor genes are often considered to be recessive.
      Alternatively, tumor progression can be influenced by functional
      haploinsufficiency. According to this conception, a disease state can
      emerge if a cell / organism has only one functional copy of a given gene
      and if it cannot  produce enough of a gene product to establish a
      wild-type condition.

      \textbf{APC:} Adenomatous Polyposis Coli (APC) gene codes for a 312 kDa
      protein. This multi--domain protein has binding sites for microtubules,
      cytoskeletal regulator proteins (IQGAP1, EB1) and Wnt signaling proteins
      (b -catenin, axin). 60\% of APC mutations in cancer present a
      C-terminal truncation, resulting in a loss of b--catenin and
      microtubule binding sites. Wnt signaling regulates, amongst others, cell
      migration, polarity, differentiation, adhesion and apoptosis. In the
      canonical Wnt signaling pathway, a destruction complex, including axin,
      GSK3, CK1a, PP2A and APC, leads to b--catenin phosphorylation,
      followed by ubiquitination, marking it for degradation in the proteasome.
      Additionally, transcription factors of the TCF/LEF family form a complex
      with factors such as Groucho and histone deacetylases. This complex binds
      to Wnt signaling target genes and thereby represses gene expression. Once
      Wnt binds to the N-terminus of a cell surface receptor of the Frizzled
      family of receptors and and a co-receptor of the LRP5/6 family, the
      destruction complex is inhibited. Consequently, b--catenin is no
      longer marked for degradation and can translocate to the nucleus. There it
      displaces the factors binding to TCF/LEF and forms a complex with TCF/LEF,
      leading to activation of gene expression of target genes. Loss or
      dysfunction of APC leads to b--catenin accumulation in the nucleus
      even in the absence of  an extracellular stimulus. APC mutations are suspected to be the initiating event in many CRCs. APC
      mutations are sufficient for the growth of benign colorectal tumors.

      \textbf{TP53:} TP53 is one of the master guardians of the genome. In
      normal situations, p53, the protein encoded by TP53, is regulated by MDM2,
      MDM4 and E3--ubiquitin ligase, which target p53 for ubiquitination and
      degradation in the proteasome. In case of cellular stress, p53 is no
      longer ubiquitinated. p53 becomes activated in several situations, which
      include  DNA damage, osmotic shock, oxidative stress or oncogene
      expression. In such situations, p53 can then stop the cell cycle at the
      G1/S and G2/M transitions, induce DNA repair, and induce apoptosis if the
      damage cannot be repaired. TP53 thereby maintains genomic stability.

      One mechanism by which p53 acts on cell-cycle arrest is by activating
      expression of p21. p21 binds to the G1/S transition complex, formed by
      CDK4/CDK6, CDK2, CDK1) and inhibits its activity, leading to cell-cycle
      arrest. Inactivation or mutation of TP53 is a crucial step in many
      cancers. A defective p53 does not bind efficiently to DNA, resulting in
      less p21 expression. As a consequence, p21 cannot act as a cell-cycle stop
      signal.

      The importance of TP53 as tumor suppressor gene becomes evident in the
      autosomal dominant Li--Fraumeni syndrome. People suffering from this disorder inherit only
      one functional copy of TP53 and are likely to develop cancer in early
      ages.

      TP53 mutations are found in 50\% of CRCs,
      especially those associated with the methylator phenotype and
      microsatellite instability. Alterations in TP53 are associated with poor
      prognosis if treated with chemotherapy. In fact, wild-type TP53 is
      required for treatment with chemotherapy based on 5-fluoroacil.

      TP53 alterations are the early events in lung carcinogenesis

      \paragraph{Oncogenes}

      Oncogenes comprise several GTPases, transcription factors, receptor tyrosine
      kinases and growth factors. Overexpressed or overactive versions of these
      proteins often lead to increased mitogenic signals, causing increased
      cell growth or proliferation. Mutations in proto--oncogenes can cause a
      loss of regulation or overactive proteins. Gene duplications or other
      chromosomal alterations lead to increased protein synthesis. Other
      mechanisms of importance include post-transcriptional mechanisms as
      misregulation of protein expression or increase of mRNA / protein stability.

      In the RAS-RAF-MAPK pathway, GRB2 binds to Tyr1068 of EGFR through their
      SH2 domain and recruits SOS, a guanine nucleotide exchange factor. Grb2
      and SOS then form a complex with the activated EGFR, which activates SOS.
      Activates SOS, through its GEF activity, then induces GDP removal from Ras
      proteins, which can subsequently bind GTP and become active. Ras then
      activates Raf serine/threonine kinase proteins, which phosphorylate and
      thereby activate MEKs, which are tyrosine/threonine kinases. Activated
      MEKS then phosphorylate and activate MAPKs, also serine/threonine kinases.
      MAPKs then act on the expression of target genes that promote cell survival,
      cell cycle progression and proliferation.

  \subsection{Targeting the EGFR signaling pathway}

    \subsubsection{Molecular Profiling of Solid Tumors}

      \textbf{Lung cancer} is the most common cancer worldwide, both in terms of
      new cases (1.8 million) and deaths (1.6 million). Smoking is a widely
      accepted risk factor, as chemical carcinogens in tobacco smoke induce
      several genetic mutations. Oncogenic triggers cause cells of the normal
      bronchial epithelium to proliferate, giving rise to meta--, hyper-- and
      dysplastic epithelial lesions. Classic diagnosis and treatment decisions
      have relied on histological analysis of the tumor. Lung cancer can be
      divided into two histological subtypes: small-cell lung cancer (SCLL) and
      non-small cell lung cancer (NSCLC). Over the last decade, it has become
      clear that these subtypes can be classified into  additional classes by
      the mutational status of recurrent driver mutations in genes that are
      frequently mutated in this type of cancer. Mutations in several oncogenes
      can be found at each stage of NSCLC and in all histological types, e.g.
      large cell carcinoma, adenocarcinoma, squamous cell  carcinoma (SCC), in
      smokers and never-smokers as well. Frequent  mutations in NSCLC affect
      EGFR (10--35\%), KRAS (15--25\%), PTEN (4--8\%), HER2 (2--4\%), DDR2
      (4\%), PIK3CA (1--3\%), BRAF (1--3\%), AKT1 (1\%), MEK (1\%) and NRAS
      (1\%). Additionally, rearrangement of ALK (3--7\%), RET (1\%) and ROS1
      (1\%) and amplifications of FGFR1 (20\%) and MET (2--4\%) are found
      recursively. These mutations are rarely observed together in the same
      tumor.

      Prognosis

      \textbf{Melanoma} develops from the malignant transformation of melanocytes
      in the basal epidermal layer of the skin. Melanoma incidence has exploded
      over the last four decades, with a 15-fold increase in the United States.
      Both genetic predisposition and environmental factors influence the risk
      of getting melanoma. Skin cancer often affects fair-skinned individuals.
      Exposure to UV light, immunosuppression and multiple nevi are risk factors.
      UV radiation causes cyclobutane pyrimidine dimers (CPDs). By joining
      adjacent pyrimidine bases, T--T, C--C or C--T dimers (UV
      fingerprints) are formed, leading to direct DNA damage. People diagnosed
      with rare genetic disorders like xeroderma pigmentosum are at great risk.
      Traditionally, melanoma has been classified based on histological and
      pathological properties, such as the thickness of the tumor, ulceration
      or the anatomic location of the tumor. Several mutations in oncogenes
      are recurrently found in melanoma, but are rarely found together in the
      same tumor. The occurrence of the different mutations differs by the
      anatomic location of the tumor, e.g. whether the specific body part is
      chronically exposed to the sun. Mutations frequently found in melanoma
      occur on BRAF (37--50\%), NRAS (13--25\%), MEK (6--7\%), NF1 (11.9\%),
      CTNNB1 (2-4\%), GNAQ (1.3\%) and GNA11 (1.2\%).

      Prognosis

      \textbf{Colorectal cancer (CRC)} 1.4 million cases are detected yearly
      with 694,000 deaths. CRC is one of the best studied cancers. The
      development of colorectal adenocarcinomas occurs over many years. Caused
      by the acquisition and accumulation of driver mutations, a normal
      colorectal epithelium can progress to adenoma, which develops into
      carcinoma, which can eventually mestastasize. Mutations recursively
      detected in CRC occur on KRAS (36--40\%), SMAD4 (10--35\%), PIK3CA
      (10--30\%), BRAF (8--15\%), PTEN (5--14\%), NRAS (1--6\%), and  AKT1
      (1--6\%).

      Prognosis

    \subsubsection{Role of EGFR in Solid Tumors}

      EGFR is a protein of the tyrosine kinase receptor family. It is anchored in
      the cytoplasmic membrane and is composed of an intracytoplasmic tyrosine
      kinase domain, a short hydrophobic transmembrane domain and an extracellular
      ligand-binding domain. Ligand binding causes a conformational change if the
      receptor, which leads to homo-- or heterodimerization, followed by an auto--
      and cross--phosphorylation of key tyrosine residues on its cytoplasmic
      domain. This forms docking sites for cytoplasmic proteins that contain
      phosphotyrosine-binding and Src homology 2 domains. These proteins are
      adaptor molecules for the RAS-RAF-MAPK and PI3K pathways. EGFR signaling
      thereby mediates cell survival, differentiation, proliferation, angiogenesis
      and migration.

      Dysregulation of the EGFR signaling pathway has been associated with the
      development and progression of solid tumors. Abnormal functioning of this
      pathway enables several cancer hallmarks such as the inhibition of
      apoptosis, induction of angiogenesis, cell-cycle progression, as well as
      promotion of cell motility and metastasis.

      TGFa activation of EGFR results in a positive growth stimulus (11, 12).
      High levels of TGFa have been associated with neoplastic transformation
      (13, 14) and overexpression of TGFa by stable trans- fection transforms
      fibroblasts in culture (15). Transgenic mice over- expressing TGFa develop
      malignant tumors at a number of sites (16–18). Furthermore, in vivo
      studies have shown that the overex- pression of TGFa enhances
      oncogene-induced carcinogenesis (19–21).

      \paragraph{EGFR expression and mutations}

      The Epithelial Growth Factor Receptor (EGFR) is overexpressed in several
      cancers. 40--80\% of NSCLC overexpress EGFR. Additionally, EGFR levels are
      higher in late stage NSCLC cells than in early stage patients. Even though
      EGFR overexpression is an important process in the development and
      progression of several cancers, its role as prognostic marker is debated.
      Some studies have found no correlation between EGFR expression and
      survival, while other studies have reported a correlation between EGFR
      expression and tumor invasiveness and poor survival.

      \paragraph{Cell survival enabled by EGFR}

      \paragraph{Cell death induced by EGFR blockade}

      http://theoncologist.alphamedpress.org/content/11/4/358.full

    \subsubsection{EGFR-targeted drugs}

      The observation that many tumors modify the EGFR signaling pathway for
      their purposes has lead to the development of EGFR-targeted agents. These
      drugs have met an enormous success, especially in combination with
      traditional cancer therapies as chemotherapy or radiation therapy.
      Interestingly, clinical trials have revealed that these agents have
      surprisingly low adverse systemic side effects. It has been proposed that
      EGFR signaling is the rate limiting factor in cancer cells, but not in
      normal cells. This might be explained by the fact that normal epithelial
      cells present normal functional cell--cell and cell--matrix contacts. This
      lessens their dependence on other survival signals such as EGFR signaling
      induced by EGF or TGFa. In contrast, cancer cells are provided with
      inadequate contacts and are thereby more dependent on other survival
      signals. Cancer cells are thereby often more susceptible to cell death
      upon EGFR blockade than normal cells.

      http://cancerres.aacrjournals.org/content/63/1/1.short

      Pharmacologically, these anti-EGFR agents can be divided into two classes.
      (i) Anti-EGFR monoclonal antibodies
      (ii) EGFR-specific tyrosine kinase inhibitors.

    \subsubsection{Predictive markers}

      Molecular testing of some predictive biomarkers is necessary to identify
      the potential success of anti-EGFR therapy. Table XXX shows a selection of
      EGFR-targeted agents, their molecular targets and FDA-approved indications.
      In fact, the mutational status of EGFR and downstream proteins is
      predictive of the successfulness of the administration of these drugs.

      Wild-type status of EGFR, KRAS, NRAS and BRAF has been associated with
      increased sensitivity to anti-EGFR antibodies. EGFR-specific tyrosine
      kinase inhibitors comprise three classes that include first generation
      reversible  EGFR-inhibitors (gefitinib, erlotinib), second generation
      irreversible inhibitors (afatinib, dacomitinib, neratinib) and third
      generation mutant-selective inhibitors (brigatinib, osimertinib,
      rociletinib). Third class agents have a better sensitivity against mutated
      than wild-type EGFR and have been designed to further decrease
      treatment-associated side effects.
      \paragraph{EGFR} is a strong predictive biomarker for the success of the
      administration of EGFR-specific tyrosine kinase inhibitors.
      EGFR activating mutations are observed in 10--35\% of NSCLC.
      90\% of these mutations are exon 19 deletions and exon 21 L585R (c.2573T>G)
      point mutations. Amongst EGFR-mutated NSCLC, EGFR L858R occurs with a
      frequency of 43\%.
      \paragraph{KRAS} mutations in CRC are found with a frequency of 36--40\%. Amongst
      these, the G12C variant is the most common (7.9\%). Critical mutations
      in the KRAS gene include variants in codons 12, 13 and 61. These mutations
      lock KRAS in its GTP-bound state, resulting in a constantly active protein.
      This then leads to a constantly active signal transduction. Blocking
      EGFR in that case is useless, as KRAS acts downstream of EGFR. Several
      KRAS point mutations in codons 12, 13 and 61 have been shown to confer
      reduced sensitivity to EGFR-targeted monoclonal antibodies in CRC. The
      situation is similar in the case of the KRAS-isoform NRAS.
      \paragraph{BRAF} mutations are very common in melanoma (37--50\%). Amongst
      BRAF-mutated melanomas, the V600E variant is found in 80--90\% cases.
      This variant occurs in the activation segment of the BRAF kinase domain
      and results in increased kinase activity. BRAF V600E mutations
      have been associated with increased sensitivity to BRAF and MEK inhibitors.


      \begin{table}[!htbp]
        \caption[Targeted Cancer Agents]{FDA-approved cancer drugs for solid tumor treatment that target the EGFR pathway}
        \centering
          \begin{tabular}{ |p{4cm}|p{3.7cm}|p{6.3cm}|}
          \hline
          Agent & Target(s) & FDA-approved indication(s) \\ \hline \hline
          Afatinib (Gilotrif) & EGFR & NSCLC (with EGFR del19 or L858R) \\
          Cetuximab (Erbitux) & EGFR & Colorectal cancer (KRAS WT) \\
          Cobimetinib (Cotellic) & MEK & Melanoma (with BRAF V600E or V600K \\
          Dabrafenib (Tafinlar) & BRAF & Melanoma (with BRAF V600 mutation) \\
          Erlotinib (Tarceva) & EGFR & NSCLC \\
          Gefitinib (Iressa) & EGFR & NSCLC (with EGFR del19 or L858R) \\
          Necitumumab (Portrazza) & EGFR & Squamous NSCLC \\
          Osimertinib (Tagrisso) & EGFR & NSCLC (with EGFR T790M) \\
          Panitumumab (Vectibix) & EGFR & Colorectal cancer (KRAS WT) \\
          Trametinib (Mekinist) & MEK & Melanoma (with BRAF V600) \\
          Vemurafenib (Zelboraf) & BRAF & Melanoma (with BRAF V600) \\
          \hline
        \end{tabular}
      \end{table}

  \subsection{Tumor DNA Sequencing}

    \subsubsection{Targeted NGS}

      \paragraph{Target enrichment methods}

    \subsubsection{Practical implications in the laboratory}

      FFPE, degradation, C>T variants

      Availability of samples

      Cancer heterogeneity

      Sequencing artifacts vs low-frequency mutations

      Bioinformatic pipeline




    \subsubsection{Genomic Instability in Cancers}


      \textbf{Microsatellite instability (MSI)}

      \textbf{Nucleotide instability (NIN)}

    \subsubsection{Tumor Suppressors and Oncogenes in Solid Tumors}

      Genomic instability in cancerous cells becomes a critical mechanism if it
      affects oncogenes or tumor suppressor genes. $
      Oncogenes comprise several GTPases, transcription factors, receptor tyrosine
      kinases and growth factors. Overexpressed or overactive versions of these
      proteins often lead to increased mitogenic signals, causing increased
      cell growth or proliferation. Mutations in proto--oncogenes can cause a
      loss of regulation or overactive proteins. Gene duplications or other
      chromosomal alterations lead to increased protein synthesis. Other
      mechanisms of importance include post-transcriptional mechanisms as
      misregulation of protein expression or increase of mRNA / protein stability.

      \textbf{APC:} Adenomatous Polyposis Coli (APC) gene codes for a 312 kDa
      protein. This multi--domain protein has binding sites for microtubules,
      cytoskeletal regulator proteins (IQGAP1, EB1) and Wnt signaling proteins
      (b -catenin, axin). 60\% of APC mutations in cancer present a
      C-terminal truncation, resulting in a loss of b--catenin and
      microtubule binding sites. Wnt signaling regulates, amongst others, cell
      migration, polarity, differentiation, adhesion and apoptosis. In the
      canonical Wnt signaling pathway, a destruction complex, including axin,
      GSK3, CK1a, PP2A and APC, leads to b--catenin phosphorylation,
      followed by ubiquitination, marking it for degradation in the proteasome.
      Additionally, transcription factors of the TCF/LEF family form a complex
      with factors such as Groucho and histone deacetylases. This complex binds
      to Wnt signaling target genes and thereby represses gene expression. Once
      Wnt binds to the N-terminus of a cell surface receptor of the Frizzled
      family of receptors and and a co-receptor of the LRP5/6 family, the
      destruction complex is inhibited. Consequently, b--catenin is no
      longer marked for degradation and can translocate to the nucleus. There it
      displaces the factors binding to TCF/LEF and forms a complex with TCF/LEF,
      leading to activation of gene expression of target genes. Loss or
      dysfunction of APC leads to b--catenin accumulation in the nucleus
      even in the absence of  an extracellular stimulus.

      APC mutations are suspected to be the initiating event in many CRCs. APC
      mutations are sufficient for the growth of benign colorectal tumors.

      http://www.wormbook.org/chapters/www_wntsignaling/wntsignaling.html
      http://www.ncbi.nlm.nih.gov/pubmed/15768032
      http://jcs.biologists.org/content/120/19/3327.long
      http://www.ncbi.nlm.nih.gov/pmc/articles/PMC2634250/

      Also in melanoma: http://www.ncbi.nlm.nih.gov/pubmed/15133491

      Present, but less in NSCLC: http://www.ncbi.nlm.nih.gov/pubmed/15072829

      \textbf{TP53:} TP53 is one of the master guardians of the genome. In
      normal situations, p53, the protein encoded by TP53, is regulated by MDM2,
      MDM4 and E3--ubiquitin ligase, which target p53 for ubiquitination and
      degradation in the proteasome. In case of cellular stress, p53 is no
      longer ubiquitinated. p53 becomes activated in several situations, which
      include  DNA damage, osmotic shock, oxidative stress or oncogene
      expression. In such situations, p53 can then stop the cell cycle at the
      G1/S and G2/M transitions, induce DNA repair, and induce apoptosis if the
      damage cannot be repaired. TP53 thereby maintains genomic stability.

      One mechanism by which p53 acts on cell-cycle arrest is by activating
      expression of p21. p21 binds to the G1/S transition complex, formed by
      CDK4/CDK6, CDK2, CDK1) and inhibits its activity, leading to cell-cycle
      arrest. Inactivation or mutation of TP53 is a crucial step in many
      cancers. A defective p53 does not bind efficiently to DNA, resulting in
      less p21 expression. As a consequence, p21 cannot act as a cell-cycle stop
      signal.

      The importance of TP53 as tumor suppressor gene becomes evident in the
      autosomal dominant Li--Fraumeni syndrome. People suffering from this disorder inherit only
      one functional copy of TP53 and are likely to develop cancer in early
      ages.

      TP53 mutations are found in 50\% of CRCs,
      especially those associated with the methylator phenotype and
      microsatellite instability. Alterations in TP53 are associated with poor
      prognosis if treated with chemotherapy. In fact, wild-type TP53 is
      required for treatment with chemotherapy based on 5-fluoroacil.

      TP53 alterations are the early events in lung carcinogenesis


      \textbf{TGF--b:}
      Melanoma: http://www.ncbi.nlm.nih.gov/pmc/articles/PMC3662904/
      http://www.ncbi.nlm.nih.gov/pubmed/18426405
      http://www.ncbi.nlm.nih.gov/pubmed/21619542

      NSCLC: http://www.ncbi.nlm.nih.gov/pubmed/20107423
      http://link.springer.com/chapter/10.1007/978-1-4419-6615-5_28#page-1
      http://www.nature.com/cdd/journal/v21/n8/full/cdd201438a.html

      Colorectal: http://www.ncbi.nlm.nih.gov/pubmed/20517689
      http://www.ncbi.nlm.nih.gov/pmc/articles/PMC3512565/
      http://hmg.oxfordjournals.org/content/16/R1/R14.full

      \paragraph{EGFR signaling pathway}: EGFR is a protein of the tyrosine kinase
      receptor family. It is anchored in the cytoplasmic membrane and is
      composed of an intracytoplasmic tyrosine kinase domain, a short
      hydrophobic transmembrane domain and an extracellular ligand-binding
      domain. Ligand binding causes a conformational change if the receptor,
      which leads to homo-- or heterodimerization, followed by an auto-- and
      cross--phosphorylation of key tyrosine residues on its cytoplasmic domain.
      This forms docking sites for cytoplasmic proteins that contain
      phosphotyrosine-binding and Src homology 2 domains. These proteins are
      adaptor molecules for the RAS-RAF-MAPK and PI3K pathways. Both pathways
      lead to cell survival, proliferation and invasion.

      PTEN/PI3K/AKT leads to cell growth, proliferation and survival
      http://www.nature.com/onc/journal/v27/n41/full/onc2008247a.html
      http://www.ncbi.nlm.nih.gov/pmc/articles/PMC3092286/
      http://www.hindawi.com/journals/isrn/2013/472432/

PIK3CA_Colorectal.pdf

      Phospholipase C

      STATs

      Src

      In the RAS-RAF-MAPK pathway, GRB2 binds to Tyr1068 of EGFR through their
      SH2 domain and recruits SOS, a guanine nucleotide exchange factor. Grb2
      and SOS then form a complex with the activated EGFR, which activates SOS.
      Activates SOS, through its GEF activity, then induces GDP removal from Ras
      proteins, which can subsequently bind GTP and become active. Ras then
      activates Raf serine/threonine kinase proteins, which phosphorylate and
      thereby activate MEKs, which are tyrosine/threonine kinases. Activated
      MEKS then phosphorylate and activate MAPKs, also serine/threonine kinases.
      MAPKs then act on the expression of target genes that promote cell survival,
      cell cycle progression and proliferation.

      http://www.ncbi.nlm.nih.gov/pmc/articles/PMC3457779/

      The RAS-RAF-MAPK pathway is deregulated in many cancers, mainly through
      activating mutatios on RAS or RAF.

      \textbf{KRAS}
      KRAS has gained interest as negative predictive
      marker of the successfulness of anti-EGFR targeted therapy. KRAS is
      mutated in 36--40\% of CRCs, 15--25\% of NSCLCs and 2\% of melanomas.
      Single nucleotide point mutations in codons 12, 13 and 61 can act as
      activating mutations. Proteins affected by such mutations are locked in
      their active GTP-bound state and are consequently constantly active.

      \textbf{BRAF}
      Apart from RAS proteins, RAF proteins are of importance in solid tumors.

      https://www.moffitt.org/File%20Library/Main%20Nav/Research%20and%20Clinical%20Trials/Cancer%20Control%20Journal/v14n3/295.pdf
      http://www.sciencedirect.com/science/article/pii/S0014579301021664
      http://www.ncbi.nlm.nih.gov/pubmed/18038764
      http://cancerres.aacrjournals.org/content/63/1/1.long

      Many cancers have been shown to be dependent on EGFR signaling.Targeting
      the EGFR signaling pathway is an attractive target blabla and has been of
      benefit in solid tumors, e.g. melanoma, CRC and NSCLC. Advantages have
      been made better survival rates. shutting egfr down -> apoptosis. but 2
      problems: resistances & not all mutations are actionable. therefore:
      molecular testing. the more comprehensible, the better. classical
      approaches only target some hotspot regions. NGS has the potential to give
      really deep insights

      In recent years, several EGFR targeted anti-cancer drugs have been
      approved by the FDA. Anti--EGFR targeted monoclonal antibodies and
      EGFR--specific tyrosine--kinase inhibitors have shown their usefulness in
      the treatment of solid tumors.

      However, solid tumors have a tendency to harbor mutations in proteins
      acting in  the EGFR signaling pathway. Table XXX shows the frequency of
      tumors  harboring mutations in EGFR or downstream proteins. Identifying
      the mutational status of these proteins is of utmost importance in
      targeted anti-EGFR therapy. Wild-type or mutated proteins provide either
      increased sensitivity or resistance to the treatment.


      Targeted cancer therapy holds the promise of highly selective
      tumor cell killing while sparing most of normal proliferating cells,
      thus avoiding some side effects of conventional cytotoxic therapy.

    \begin{table}[!htbp]
        \caption[Occurrence of mutations]{EGFR signaling pathway components affected in colorectal cancer, melanoma and non-small cell lung carcinoma}
        \centering
        \begin{tabular}{ |p{2cm}|p{2cm}|p{2cm}|p{2cm}|}
        \hline
        Gene & CRC (\%) & Melanoma & NSCLC \\ \hline \hline
        EGFR & NA & NA & 10--35 \\
        KRAS & 36--40 & 2 & 15--25 \\
        NRAS & 1--6 & 13--25 & 1 \\
        BRAF & 8--15 & 37--50 & 1--3 \\
        PTEN & 5--14 & NA & 4--8 \\
        PIK3CA & 10--30 & & 1--3 \\
        \hline
      \end{tabular}
    \end{table}
(http://www.nature.com/bjc/journal/v112/n2/full/bjc2014476a.html)

    \begin{table}[!htbp]
        \caption[Targeted Cancer Agents]{FDA-approved cancer drugs for solid tumor treatment that target the EGFR pathway}
        \centering
        \begin{tabular}{ |p{4cm}|p{3.7cm}|p{6.3cm}|}
        \hline
        Agent & Target(s) & FDA-approved indication(s) \\ \hline \hline
        Afatinib (Gilotrif) & EGFR & NSCLC (with EGFR del19 or L858R) \\
        Cetuximab (Erbitux) & EGFR & Colorectal cancer (KRAS WT) \\
        Cobimetinib (Cotellic) & MEK & Melanoma (with BRAF V600E or V600K \\
        Dabrafenib (Tafinlar) & BRAF & Melanoma (with BRAF V600 mutation) \\
        Erlotinib (Tarceva) & EGFR & NSCLC \\
        Gefitinib (Iressa) & EGFR & NSCLC (with EGFR del19 or L858R) \\
        Necitumumab (Portrazza) & EGFR & Squamous NSCLC \\
        Osimertinib (Tagrisso) & EGFR & NSCLC (with EGFR T790M) \\
        Panitumumab (Vectibix) & EGFR & Colorectal cancer (KRAS WT) \\
        Trametinib (Mekinist) & MEK & Melanoma (with BRAF V600) \\
        Vemurafenib (Zelboraf) & BRAF & Melanoma (with BRAF V600) \\
        \hline
      \end{tabular}
    \end{table}

  \subsection{Targeted Sequencing}

    \subsubsection{Target Enrichment Methods}

    \subsubsection{Illumina Sequencing Chemistry}

  \subsection{NGS Data Analysis}

    \subsubsection{GATK Best Practices}

  \subsection{Practical Implications in the Laboratory}

  \subsection{Aims of the Thesis}
